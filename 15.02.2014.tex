\section{Політологія як наука і суспільне явище}
Питання лекції:
\begin{itemize}
\item Теоретичні засади політології: поняття, зміст, структура, функції та методи;
\item Політика - як одна із категорій політології;
\item Суб’єкти і об’єкти політики.
\end{itemize}

1914 рік із 324 університетів в 200 викладалася політологія як окрема дисципліна.

В 1948 році у Парижі пройшов міжнародний конгрес політологів. Вони вирішили створити свій власний курс, оскільки після світових війн питання політики гостро стало для людства.

В 1949 році було прийнято загальносвітове рішення про те, що в усіх відомих університетів вводиться додатковий курс політології.

Вивчатися буде: історія політичної думки, функціонування державних установ, порядок створення і функціонування політичних партій та рухів, вивчення міжнародних відносин.

{\bf Політика} в перекладі з давньогрецької означає: місто або держава, мистецтво керувати державою, конституція. Політика є однією із основних %як неочікувано, ну правда (сарказм)
категорій політології.

Політологія розглядає три послідовні сфери, в яких політика являє інтерес для політолога. А саме:
\begin{itemize}
\item Політика, що визначається як курс, на основі якого приймаються рішення, заходи, щодо формування і вдосконалення завдань;
\item Політика розуміється як конкретна сфера, де окремі люди і політичні утворення ведуть боротьбу за здобуття політичної влади;
\item Політика розглядається більш широко, а саме як мистецтво керувати людьми.
\end{itemize}
%Можливо, розбити на пункти

{\bf Політика} - це сфера виявлення інтересів соціальних груп, їх зіткнення і протиборство; спосіб певної субординації цих інтересів, підпорядкованих їх найвищому началу, значущому та обов’язковому; рух соціальних груп, спільнот, які прагнуть реалізувати свої інтереси через загальний інтерес, що набуває примусової форми для решти суспільства; засіб єдності і цілісності суспільства; становлення людини як вільної, неповторної та унікальної.

Принципи політики:
\begin{enumerate}
\item Єдність соціально-класового і загальнолюдського;
\item Принцип демократизму;
\item Принцип гуманізму;
\item Моральність політики.
\end{enumerate}

{\bf Політологія} - це наука про політику та її взаємовідносини з людиною, суспільством та державою; це певна система знань про політичні системи, міжнародні відносини, управління соціальними процесами, політичною ідеологією та історією політичних вчень.

Предметом політології є феномен політичної влади, закономірності її функціонування, розвитку і використання в суспільстві.

Основні функції політології:
\begin{itemize}
\item {\bf Теоретико-пізнавальна}, що направлена на вивчення, систематизацію, пояснення, аналіз, узагальнення та оцінку політичних явищ;
\item {\bf Методологічна}, що охоплює способи, методи і принципи теоретичного дослідження політичної сфери і практичної реалізації здобутих знань;
\item {\bf Аналітична}, що полягає у накопиченні, вивченні, систематизації фактів та явищ політичного життя;
\item {\bf Прогностична}, що полягає у передбачені шляхів розвитку політичних процесів, моделюванні різних варіантів політичної поведінки;
\item {\bf Функція політичної соціалізації}, що пов’язана із включенням кожної людини в політичну сферу життя суспільства та формування політичної культури;
\item {\bf Світоглядна}.
\end{itemize}

Функції то функції, а ще методи, якими можна кожну функцію реалізувати. Основні методи політології:
\begin{itemize}
\item {\bf Історичний}, що полягає у вивчені політичних процесів та явищ з точки зору історичного розвитку;
\item {\bf Інституційний}, що включає в себе вивчення інститутів, за допомогою яких і здійснюється політична діяльність (держава, партії, рухи, об’єднання тощо);
\item {\bf Соціальний}, що направлений на з’ясування соціальної зумовленості політичних явищ;
\item {\bf Емпіричний} (прикладний), що досліджує політичну діяльність, шляхом застосування відповідних засобів аналізу та дослідження;
\item {\bf Системний}, що забезпечує сприйняття всіх елементів політичної системи та їх дослідження у відповідних взаємозв’язках в межах цілого;
\item {\bf Структурно-функціональний}, що направлений на аналіз вивчення змін суспільства та соціальних змін окремих індивідів;
\item {\bf Соціально-психологічний} (біхевіористичний), що орієнтований на вивчення поведінки груп, мас, особистостей, що виконують будь-яку політичну діяльність, спрямовану на досягнення тієї чи іншої політичної мети;
\item {\bf Порівняльний}, що заснований на порівнянні тих чи інших політичних явищ, а саме політичних режимів, державного устрою тощо. Цей метод дає можливість встановити спільні і відмінні риси політичного життя різних епох, країн, народів;
\item {\bf Політичне моделювання}, що припускає можливе передбачення розвитку політичних подій, на підставі яких можна прийняти ефективне політичне рішення;
\item {\bf Антропологічний}, що зумовлений необхідністю вивчення природи людського роду.
\end{itemize}
