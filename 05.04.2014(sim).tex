\section{Політичний режим: поняття, типології та їх зміст. Демократичний політичний режим  та  проблеми його формування  в Україні}
\subsection{План семінару}
\begin{enumerate}
\item Зміст поняття “ політичний режим”,  його типології .
\item Демократичний  політичний режим та його ознаки.
\item Проблеми формування демократичного політичного режиму в Україні.
\end{enumerate}

\subsection{Зміст поняття “політичний режим”, його типології}
\textbf{Політичний режим} — тип, характер влади в країні; сукупність засобів і методів здійснення політичної влади, яка відображає характер взаємовідносин громадян і держави.

Політичний режим визначається способом і характером формування представницьких установ, органів влади, співвідношенням законодавчої, виконавчої і судової влади, центральних і місцевих органів, становищем, роллю та умовами діяльності громадських організацій, рухів, партій, правовим статусом особи, ступенем розвитку демократичних свобод.

\noindent\textbf{Типологія:}
Політичні режими можна розрізнити за двома головними критеріями — джерелом влади та межами цієї влади.

\noindent\textbf{За джерелом влади:}
\begin{itemize}
\item\textbf{Демократичні режими} \item форма державно-політичного устрою суспільства, в якій народ виступає джерелом влади на принципах рівності, свободи і солідарності. Зовнішніми ознаками демократичного режиму є багатопартійність, наявність представницьких органів, формальне визнання народу джерелом влади, визнання права всіх громадян на участь у формуванні органів державної влади, контроль за їхньою діяльністю, вплив на прийняття спільних для всіх рішень на засадах загального, рівного виборчого права і здійснення цього права у процедурах виборів, референдумів тощо, переважне право більшості при прийнятті рішень, чітке регламентування політичних процедур та процесів.
\item\textbf{Недемократичні режими} \item форма державно-політичного устрою суспільства, або спосіб правління, оснований на владі авторитету, політико-правовій нерівності соціальних груп та прошарків суспільства, використанні насильства. Втім, авторитарні режими можуть будуватись як на базі авторитету звичаю, традиції (монархії), так і на авторитеті сили (диктатури). Зовнішними ознаками авторитарних режимів є відсутність або формальний характер представницьких органів влади, відмова від принципу поділу влади, різний політико-правовий статус окремих соціальних груп, і в зв’язку з цим – нерівність (або взагалі відсутність) виборів, інколи – посилення ролі армії та інших силових структур. Різновиди авторитарних режимів: традиціоністські та модернізаторські; популістські, націоналістичні, корпоративістські, військові, авторитарно-бюрократичні.

\end{itemize}
\noindent\textbf{За критерієм меж влади:}
\begin{itemize}
\item \textbf{Ліберальні режими} - організація політичної системи, в якій влада держави обмежена сферою невід’ємних прав і свобод особистості. Це режим, в якому досить розвинутим є громадянське суспільство, різні самодіяльні громадянські ініціативи, тобто організації, які незалежні від держави, гарантуються основні права та свободи громадян.
\item \textbf{Тоталітарні режими} - політична система, яка намагається — заради тієї чи іншої мети — повністю (тотально) контролювати все життя суспільства в цілому і кожної людини окремо. Поняття “тоталітарна держава” вперше було застосоване італійським диктатором Беніто Муссоліні, причому в позитивному контексті. Тоталітарні режими почали активно вивчатись після Другої світової війни через злочинні явища фашизму, нацизму та сталінізму.
\end{itemize}

Іноді додатково до основних типів виділяють такі різновиди режимів як диктаторські, фашистські, екстремістські, парламентські, президентські, монархічні, республіканські, надзвичайного правління, абсолютистські і т.інше.
\section{Демократичний  політичний режим та його ознаки}
Поняття “демократія” багатогранне. Його використовують на позначення типу політичної культури, певних політичних цінностей, політичного режиму. У вузькому розумінні “демократія” має тільки політичну спрямованість, а в широкому — це форма внутрішнього устрою будь-якої суспільної організації.

\textbf{Класичне визначення демократії} дав А. Лінкольн:
\begin{quotation}
\textit{Демократія — правління народу, обране народом, для народу.}
\end{quotation}
\textbf{Характерною особливістю} демократичного політичного режиму є децентралізація, роззосередження влади між громадянами держави з метою надання їм можливості рівномірного впливу на функціонування владних органів.

\textbf{Демократичний режим} – це форма організації суспільно-політичного життя, заснованого на принципах рівноправності його членів, періодичної виборності органів державного управління і прийняття рішень у відповідності з волею більшості.

\textbf{Основними ознаками демократичного політичного режиму} є:
\begin{itemize}
\item наявність конституції, яка закріплює повноваження органів влади й управління, механізм їх формування;
\item визначено правовий статус особистості на основі принципу рівності перед законом;
\item поділ влади на законодавчу, виконавчу та судову з визначенням функціональних прерогатив кожної з них;
\item вільна діяльність політичних і громадських організацій;
\item обов’язкова виборність органів влади;
\item розмежування державної сфери та сфери громадянського суспільства;
\item економічний та політичний, ідеологічний плюралізм (заборони торкаються лише антилюдських ідеологій).
\end{itemize}
За демократії \ul{\textit{політичні рішення}} завжди \textit{альтернативні}, \ul{\textit{законодавча процедура чітка й збалансована}}, а \ul{\textit{владні функції є допоміжними}}.

\ul{\textit{Демократії властива зміна лідерів}}. Лідерство може бути як індивідуальним, так і колективним, але завжди має раціональний характер.

Демократичний режим характеризують \ul{\textit{високий рівень суспільного самоврядування, переважаючий консенсус у відносинах між владою й суспільством}}. Одним із головних принципів демократії є \ul{\textit{багатопартійність}}. У політичному процесі завжди бере участь і опозиція, яка виробляє альтернативні політичні програми й рішення, висуває своїх Претендентів на роль лідера. \ul{\textit{Головна функція опозиції}} за демократичного політичного режиму — визначати альтернативні напрями розвитку суспільства та складати постійну конкуренцію правлячій еліті. Сутнісними ознаками демократії є \ul{\textit{електоральні}} (лат. elector — виборець) \ul{\textit{змагання}}, \ul{\textit{можливість розподілу інтересів, націленість на консолідацію суспільства}}.

За демократії держава функціонує заради громадян, а не навпаки, існують умови для подальшого розвитку громадянського суспільства. Демократія і в політичному, і в загальнолюдському розумінні є магістральним шляхом, своєрідним ідеалом майбутнього розвитку суспільства та людської цивілізації загалом.
\subsection{Проблеми формування демократичного режиму в Україні}
Розгортання конституційного процесу ускладнювалось сповільненим виробленням науково обґрунтованої моделі майбутнього суспільно-політичного устрою, протистоянням гілок влади та непримиренною боротьбою навколо питань законодавства різних політичних сил (статус Республіки Крим, державна мова, державна символіка, приватна власність, розподіл владних повноважень та ін).
Основний Закон проголосив Україну демократичною, соціальною, правовою державою. Людина, її життя і здоров'я, честь і гідність, недоторканість і безпека були визначені найвищими цінностями.
Щоб демократія стала у суспільстві новою діючою системою влади, важливо щоб люди не тільки розуміли сутність її основних принципів, а й були згідні жити згідно з цими принципами – самостійно, без постійної опіки, всесильної влади, з усією повнотою відповідальності.

Наша країна не дуже багата демократичними традиціями. По суті, протягом всього періоду радянського тоталітаризму в Україні не розвивалися елементи особистих свобод і правової держави, демократичної свідомості суспільства та особистості. На сьогоднішньому етапі важливо не прискорювати штучними засобами процес демократизації суспільного життя, проте не варто і гальмувати його схожими методами. Демократія має визрівати на національному ґрунті поступово і послідовно, її межі повинні бути обумовлені логікою розвитку посткомуністичного суспільства, його трансформацією.
Основними передумовами формування демократичного суспільства в Україні є:
\begin{itemize}
\item розширення економічної свободи;
\item радикальна зміна інститутів суспільства, усієї системи цінностей й психології людей, які породила тоталітарна система;
\item підвищення рівня політичної дисципліни і політичної культури громадян;
\item встановлення ефективного контролю суспільства над політикою можновладців;
\item подолання економічних проблем.
\end{itemize}
Динамізм демократичних процесів в Україні залежить від рівня:
\begin{itemize}
\item політичної активності громадян;
\item економічної, соціальної й політичної стабільності суспільства;
\item співвідношення політичних сил;
\item розвитку національної ідеї та правосвідомості.
\end{itemize}
Освовною проблемою для формування демократичного суспільства в Україні є те, що діюча влада (за будь якого президента) не підклувалася про те, щоб виховувати в українцях розміння принципів демократії та самі не сильно їм слідували. Відсутність гідних політичних кандидатів призводить до політичної апатії та сильно перешкоджає формуванню демократії. 
