\section{Демократична форма державної влади в суспільстві. Правова держава і громадянське суспільство та умови їх формування в Україні}
\subsection{План}
\begin{enumerate}
\item Поняття “демократична держава”, її суттєві ознаки та типологія.
\item Сутність правової держави та її ознаки.
\item Сутність громадянського суспільства та її ознаки.
\item Українська держава та тенденції її розвитку.
\end{enumerate}
\subsection{Поняття “демократична держава”, її суттєві ознаки та типологія}
\textbf{Демократична держава} - форма держави, що ґрунтується на конституційному визнанні народу джерелом влади та реальному здійсненні народної влади через рівноправну участь громадян в управлінні загальнодержавними і місцевими справами, у контролі за державною діяльністю; виборності та змінюваності вищих органів державної влади; існуванні багатопартійності; забезпеченні прав і свобод людини й меншості відповідно до міжнародних стандартів.

\noindent\textbf{Ознаки демократичної держави:}
\begin{enumerate}
\item \textit{визнання народного суверенітету} - влади народу як верховного носія і джерела державної влади;
\item \textit{рівноправна участь всього народу} (а не тільки частини населення) в управлінні справами в суспільстві і державі безпосередньо (самоврядування) і через представницькі органи. Рівний доступ усіх до державної влади передбачає участь у формуванні органів держави, контролі за їх діяльністю, підтриманні постійного контакту з ними населення;
\item \textit{демократична процедура утворення органів держави} в результаті конкурентних, вільних і чесних виборів (загальне право голосу; рівність у виборах; таємне голосування; прямі (безпосередні) вибори, що передбачають змінюваність парламенту, президента, місцевих рад демократичним шляхом. У демократичній державі одні й ті ж люди не повинні тривалий час безперервно обіймати посади в органах влади: це викликає недовіру громадян, призводить до втрати легітимності цих органів. Потрібні змінюваність, підконтрольність і взаємоконтроль, рівна можливість кожного реалізувати свої виборчі права;
\item \textit{політична свобода і рівність громадян перед законом і судом}. Політична свобода означає свободу вибору суспільного ладу, форми правління, право визначати і змінювати конституційний устрій; забезпечувати захист прав людини. Вона ґрунтується на свободі економічній, різних формах власності, рівному доступі до власності та на багатоманітності культурного життя. Свобода має первинне призначення - на її основі можуть виникнути рівність і нерівність, але вона передбачає рівноправ'я, рівність громадян перед законом і судом, незалежно від тих відмінностей, які властиві окремим індивідам, і незалежно від тих стартових можливостей і ресурсів, які кожен з них має;
\item \textit{проголошення і реальна гарантованість основоположних прав людини і громадянина}, тобто налагодженість юридичних інститутів та юридичних механізмів і процедур дієвого захисту особи від свавілля і беззаконня;
\item \textit{наявність систем самоврядування} - місцевого самоврядування як особливого виду суспільного управління на місцях, здійснюваного територіальними спільностями громадян і органів, що ними обираються; професійних організацій; виробничих, громадських об'єднань за їх інтересами та ін. Особливість систем самоврядування виявляється у поєднанні суб'єкта і об'єкта управління, у спільній участі у прийнятті і реалізації рішень, визнанні над собою влади тільки власного об'єднання;
\item \textit{багатопартійність та ідеологічний плюралізм} - наявність кількох політичних (у тому числі опозиційних) партій, різноманіття політичних думок - партійних та інших ідеологічних підходів до вирішення загальнодержавних і регіональних (місцевих) завдань. Виключаються державна цензура й ідеологічний диктат. Проте політичне й ідеологічне різноманіття має свої межі - забороняється створення політичних партій, програмні цілі і дії яких спрямовані на насильницьку зміну конституційного ладу, порушення цілісності держави, її незалежності та ін.;
\item \textit{ухвалення рішень волею більшості за обов'язкового додержання прав меншості}, тобто поєднання волі більшості з гарантіями прав і свобод особи, котра перебуває в меншості - етнічній, релігійній, політичній; відсутність будь-якої дискримінації, що означає утвердження якісно нових принципів демократичного плюралізму - демократії консенсусного суспільства, тобто демократії "вирішальної меншості" (а не диктатури більшості, як-от: парламентської);
\item \textit{взаємна відповідальність держави і громадянина}. Відповідальність держави виражається у недопущенні свавілля (з боку державних органів і їх посадовців) щодо громадянина, у вимозі утримуватися від вчинення дій, що порушують права і свободи людини, тобто у забезпеченні законності. Основним арбітром у можливих конфліктах між державою і громадянином є незалежний і демократичний суд.
\end{enumerate}
\noindent\textbf{Типологія демократичних систем}

Нижче перераховані основні типи демократичних систем.
\subsubsection{Домінуюча гілка влади}
\begin{itemize}
\item \textit{Парламентська демократія}. Уряд призначається законодавчим органом влади. Уряд і його глава (прем'єр-міністр) також можуть бути підзвітні церемоніальному главі держави (монарху, президенту чи спеціальному органу). В парламентській республіці глава держави періодично обирається парламентом, або цю посаду суміщає голова уряду.
\item \textit{Президентська республіка}. Президент вибирається народом безпосередньо і є главою виконавчої влади.
\item Існують також \textit{змішані системи}.
\end{itemize}
\subsubsection{Регіональна ієрархія влади}
\begin{itemize}
\item \textit{Унітарна держава}. Політична влада зосереджена в руках центрального уряду, який визначає обсяги владних повноважень регіональних органів влади.
\item \textit{Федерація}. Згідно з конституцією, влада поділена між центральним урядом і щодо автономними регіональними урядами.
\end{itemize}
\subsubsection{Структура законодавчої влади}
\begin{itemize}
\item \textit{Однопалатний парламент}. Нормативні акти приймаються на засіданнях за участю всіх членів парламенту.
\item \textit{Двопалатний парламент}. Законодавчі збори складаються з двох палат, які формуються і функціонують окремо. Одні нормативні акти можуть вимагати схвалення тільки однієї палати, інші — обох палат.
\end{itemize}
\subsubsection{Система виборів до представницьких органів}
\begin{itemize}
\item \textit{Мажоритарна виборча система}. Територія поділена на округи, кожен з яких має право на одного представника в законодавчих зборах. Цим депутатом стає кандидат, який набрав більшість голосів.
\item \textit{Пропорційна виборча система}. Політичні партії в законодавчих зборах отримують число місць, пропорційне числу набраних ними голосів.
\item \textit{Групова виборча система}. Певні групи населення висувають своїх депутатів згідно з заздалегідь обговореною квотою.
\end{itemize}
\subsubsection{Число провідних партій}
\begin{itemize}
\item \textit{Двопартійна система}. У політичному спектрі домінують дві великі партії.
\item \textit{Багатопартійна система}. Призначенню уряду зазвичай передує формування правлячої коаліції з двох або більше партій, представлених в законодавчих зборах.
\end{itemize}
\subsection{Сутність правової держави та її ознаки}
\textbf{Правова держава} - це держава, у якому організація й діяльність державної влади в її взаєминах з індивідами і їхніми об'єднаннями заснована на праві і йому відповідає.

Ідея правової держави спрямована на обмеження влади (чинності) держави правом; на встановлення правління законів, а не людей; на забезпечення безпеки людини в його взаємодіях з державою.

Політична влада - в будь-якій державі в цілому організується і функціонує у правовій формі, що, однак, не виключає можливості її порушення владою на окремих етапах розвитку. Це дає підстави стверджувати, що конкретні держави неоднаково обмежені правом. Започаткування правової держави означало прагнення до розбудови державності, в якій влада була б максимально обмежена правом і правами людини. Таким чином, правова держава певною мірою є поняттям ідеологічним з історично змінюваним змістом (від ліберальної до соціальної моделі). З цього випливає: те, що вважалося ідеалом на момент започаткування теорії, на початку XXI століття не повністю відповідає сучасним уявленням про роль і місце держави в житті суспільства.
\subsubsection{Основні ознаки правової держави:}
\begin{enumerate}
\item \textit{Здійснення державної влади відповідно до принципу її поділу} на законодавчу, виконавчу, судову з метою не допустити зосередження всієї повноти державної влади в або одних руках, виключити її монополізацію, узурпацію однією особою, органом, соціальною верствою, що закономірно веде до "жахаючого деспотизму" (Ш. Монтеск'є).
\item \textit{Наявність Конституційного Суду} - гаранта стабільності конституційного ладу - органа, що забезпечує конституційну законність і верховенство Конституції, відповідність їй законів і інших актів законодавчої й виконавчої влади.
\item \textit{Верховенство закону й права}, що означає: жоден орган, крім вищого представницького (законодавчого), не вправі скасовувати або змінювати прийнятий закон.Всі інші нормативно-правові акти (підзаконні) не повинні суперечити закону. У випадку ж протиріччя пріоритет належить закону.Самі закони, які можуть бути використані як форма легалізації сваволі (прямій протилежності права), повинні відповідати праву, принципам конституційного ладу. Юрисдикцією Конституційного Суду чинність неправового закону підлягає призупиненню, і він направляється в Парламент для перегляду.
\item \textit{Зв'язаність законом рівною мірою як держави в особі його органів, посадових осіб, так і громадян, їхніх об'єднань}. Держава, що видала закон, не може саме його й порушити, що протистоїть можливим проявам сваволі, свавілля, уседозволеності з боку бюрократії всіх рівнів.
\item \textit{Взаємна відповідальність держави й особистості}:
особистість відповідальна перед державою, але й держава не вільно від відповідальності перед особистістю за невиконання взятих на себе зобов'язань, за порушення норм, що надають особистостей права.
\item Реальність закріплених у законодавстві основних прав людини, прав і свобод особи, що забезпечується наявністю відповідного правового механізму їхньої реалізації, можливістю їхнього захисту найбільш ефективним способом - у судовому порядку.
\item Реальність, дієвість контролю й нагляду за здійсненням законів, інших нормативно-правових актів, слідством чого є довіра людей державним структурам, обіг для дозволу сугубо юридичних суперечок до них, а не, наприклад, у газети, на радіо й телебачення.
\item Правова культура громадян - знання ними своїх обов'язків і прав, уміння ними користуватися; поважне відношення до права, що протистоїть "правовому нігілізму" (віра в право чинності й невір'я в чинність права).
\end{enumerate}
\subsection{Сутність громадянського суспільства та її ознаки}
Поняття «громадянське суспільство», як правило, використовується в зіставленні з поняттям «держава». Вони відображають різноманітні аспекти життя суспільства, протистоячі один одному.
\subsubsection{Громадянське суспільство:}
\begin{itemize}
\item є сукупністю міжособових відносин і сімейних, суспільних, економічних, культурних, релігійних і інших структур, які розвиваються в суспільстві зовні кордоныв і без втручання держави.
\item предстає у вигляді соціального, економічного і культурного простору, в якому взаємодіють вільні індивіди, реалізовуючі приватні інтереси і здійснюючі індивідуальний вибір.
\end{itemize}
Історично громадянське суспільство прийшло на зміну традиційному, станово-кастовому, в якому держава практично співпадала з майновими класами і була відособлена від основної маси населення. Його основою є вільний індивід, незалежний від влади і форм колективного життя. Найістотніша передумова його свободи - інститут приватної власності, що формує розвинуту цивільну самосвідомість.

Відособлення суспільства від всепроникаючої влади держави завершилося в ході революцій ХУП-ХУШ століть і подальших реформ. Сам же перехід відображав формування нових соціально-економічних, політичних і культурних реальностей:
\begin{itemize}
\item постійним розвитком товарно-грошових відносин;
\item промисловою революцією;
\item появою прошарку самостійних товаровиробників;
\item кризою легітимності абсолютистських режимів;
\end{itemize}
Головними передумовами громадянського суспільства є:
\begin{itemize}
\item законодавче закріплення юридичної рівності людей на основі надідення їх правами і свободами;
\item юридична свобода людини, обумовлена матеріальним благополуччям, свободою підприємництва, наявністю приватної власності, яка є економічною основою цивільного суспільства;
\item створення механізмів саморегуляції і саморозвитку, формування сфери невладних відносин вільних індивідів, що володіють здатністю і реальною можливістю здійснювати свої природні права.
\end{itemize}
Сам термін «громадянське суспільство» використовується як в широкому, так і у вузькому значеннях. В широкому значенні громадянське суспільство включає всі соціальні структури і відносини, які безпосередньо не регулюються державою. Воно виникає і змінюється в ході природноісторичного розвитку як автономна, безпосередньо не залежна від держави сфера. При такому підході громадянське суспільство сумісно не тільки з демократією, але і авторитаризмом, і лише тоталітаризм означає його повне, а частіше часткове поглинання політичною владою.

У вузькому значенні - це суспільство на певному етапі свого розвитку, коли воно виступає соціально-економічною основою демократичної і правової держави. Сучасне розуміння громадянського суспільства в політології переважно виходить з цього вузького значення.

\textbf{Громадянське суспільство} - це сукупність між особових відносин, які розвиваються зовні кордонів і без втручання держави, а також розгалужена мережа незалежних від держави суспільних інститутів, реалізуючих індивідуальні і колективні потреби.

Головні ознаки громадянського суспільства:
\begin{itemize}
\item розмежування компетенції держави і суспільства, незалежність інститутів громадянського суспільства від держави в рамках своєї компетенції;
\item демократія і плюралізм в політичній сфері;
\item ринкова економіка, основу якої складають недержавні підприємства;
\item середній клас як соціальна основа громадянського суспільства;
\item правова держава, пріоритет прав і свобод індивіда перед інтересами держави;
\item ідеологічний і політичний плюралізм;
\item свобода слова і засобів масової інформації.
\end{itemize}
Громадянське суспільство є системою, в якій переважають горизонтальні (невладні) зв'язки і відносини, що само організовується і само розвивається. В державі ж переважаючими є вертикальні зв'язки.

\textbf{Структура громадянського суспільства}. В сучасній політології громадянське суспільство розглядається як складна і багаторівнева система невладних зв'язків і структур. Вона включає: 1) всю сукупність між особових відносин, які розвиваються зовні кордонів і без втручання держави; 2) розгалужену систему незалежних від держави суспільних інститутів, що реалізовують повсякденні індивідуальні і колективні потреби.
\subsection{Українська держава та тенденції її розвитку}
\textbf{Україна} — держава у Східній Європі, у південно-західній частині Східноєвропейської рівнини. Площа становить 603 628 км квадратних. Найбільша країна, чия територія повністю лежить в Європі, друга країна за величиною наєвропейському континенті, якщо враховувати Росію. Межує з Росією на сході і північному сході, Білоруссю на півночі,Польщею, Словаччиною та Угорщиною — на заході, Румунією та Молдовою — на південному заході. На півдні і південному сході омивається Чорним й Азовським морями.

Українська держава має досить складну та заплутану історію, яку постійно переглядають та намагаються переписати.

Основні всплески української державності:
\begin{enumerate}
\item Гальцько-Волинське князівство (1199 – 1349)
\item Гетьманьщина або Військо Запоріжське (1649 – 1764)
\item Українська державай (1918)
\item УРСР (1919 – 1991)
\item Україна (1991 - сьогодення)
\end{enumerate}
Українська держава повільно просувається до сучаcних стандартів світової цивілізації. Дуже важливо зрозуміти, де коріння серйозних прорахунків, а той помилок припустилися українські політики. Загальні фрази про пріоритети країни, проголошені впливовими політичними партіями в більшості не відповідають потребам української держави, або не виходять за рамки банального популізму. Завдання всіх політичних сил переважно полягало в тому, щоб виграти тактично і майже нічого не робилося для вибудовування реального руху стратегічної перспективи.

На протязі всієї невеликої історії незалежності, Україна, поки виглядає слабкою і політично нестабільною державою. Незважаючи на те, що Конституція України задекларувала правові і соціальні стандарти соціального життя, права і свободи громадян постійно порушуються, правові норми нехтуються роздмуханою і свавільною бюрократією. А цілісність і єдність держави розмивається самовладними  регіональними партійними лідерами і їхніми прихильниками у виконавчих структурах влади, що призводить до реальної загрози територіальної цілісності країни. Активними і безпосередніми порушниками законів країни і конституції держави виявились насамперед ті, які за своєю посадою і відповідним політичним статусом повинні були їх боронити. Але у боротьбі за владні повноваження  відбувались дії, які виходили за рамки моральної уяви цивілізованого світу, тобто використовувались будь-які брудні засоби, аби досягнути поставленої мети.

В сучасних умовах Україна не може залишатись у перманентному стані політичної, соціальної і економічної нестабільності. Реформи, що покликані перетворити Україну на сильну квітучу, демократичну державу, здатну забезпечити фундаментальні права і свободи людини і високий життєвий рівень громадян, впливову державу на міжнародній арені залишаються незмінними. Наслідком реформ повинно стати не перемога якогось бізнес-політичного проекту, а суспільний лад побудований за критеріями постіндустріального, інформаційного, високотехнологічного суспільства, що розвивається на основі досягнень сучасної науки і техніки.  Це передбачає зміцнення держави, де кожна гілка влади не поборює іншу, а діє  відповідно визначенної компетенції для забезпечення намічених цілей. Неухильно зміцнювати верховенство права, запроваджуючи ефективні форми і методи управління, відділивши великий бізнес від прийняття важливих державних рішень. Економічна роль держави повинна визначатись потужною соціальною орієнтацією поряд з створенням державою реальних умов підтримки і розвитку середнього класу, економічно незалежного від держави, здатного впливати на прийняття важливих державних рішень.

Отже, осмислюючи певний позитивний і негативний досвід у розбудові сучасної демократичної держави, слід зазначити, що сьогодні особливо актуальною залишається розробка таких ключових проблем теорії, як роль держави у підтримці, захисту і зростання середнього класу, регулюючі функції держави в економіці і соціальній сфері, поглиблені пошуку реального забезпечення прав і свобод громадян, розвитку правової системи з питань правового регулювання з ефективної боротьби з корупцією і організованою злочинністю, а також правових аспектів у пошуку України свого місця і міжнародного впливу у новому світопорядку. Це означає, “що вся влада базується на усвідомленні того, що її будуть застосовувати відповідно до загально прийнятих принципів, згідно з якими особи, наділені владою обираються через те, що всі переконані в їхній здатності робити тільки те що правильно, а не тому, що все, що вони не зробили, уважається правильним”. Треба визнати, що спрощене розуміння загальновідомої класичної правової теорії з розподілу влад, приоритету прав і свобод людини і багатьох інших проблем не може бути достатнім. Ідеальні схеми, що постали із практичного досвіду багатьох розвинутих країн, автоматично не спрацьовують у країнах перехідного періоду. Модернізація суспільних відносин вимагає в багатьох відносинах свого особливого регулювання, пристосованого до власної ментальності, культурного і історичного досвіду, у певному розумінні, потребують власного права. Таке право відрізняється  від права стабільного суспільства певними суттєвими особливостями, в першу чергу своєю динамікою, частою зміною правових норм. Нерідко сполучення різнопланових норм, змішання старого і нового, збереження норм, які у певних напрямках гальмують розвиток соціально-політичних і економічних перетворень, неминуче спотворюють прагнення і розвиток суспільства в цілому. Саме тому необхідно значну увагу треба приділяти увагу осмисленню перехідного періоду, для пом’якшення негативних наслідків.

Останні події пов’язані з грузинсько-осетинським конфліктом засвідчують про необхідність вжити ряд необхідних перетворень спрямованих на зміцнення держави. Ідея сильної держави необхідна не тільки у зовнішніх стосунках неспокійного світу, але й у внутрішніх стосунках всередині країни для успішного подолання злочинності, корупції, сепаратизму, підривної діяльності закордонних служб. Разом з тим, сильна держава не повинна асоціюватися з могутнім бюрократичним апаратом державної влади спрямованої проти будь-яких проявів незадоволення громадян на дії політичної верхівки. Зміцнення держави повинно відбуватись таким чином, щоб одночасно наповнювався зміст і реалізовувались всі принципи правової держави.
