\section{Історія розвитку політичної думки від стародавніх часів до сьогодення} 
\subsection{Питання лекції}
\begin{itemize}
\item Політично-філософські ідеї у стародавньому світі в часи Античності та Середньовіччя;
\item Політичні ідеї Відродження та часів перших буржуазних революцій 16-19 століття;
\item Політична думка епохи просвітництва. Ідея утопічності соціалістів;
\item Основні напрямки політичної думки сьогодення.
\end{itemize}
\subsection{Стародавній Китай}
Джоу Гун (11-10 століття до нашої ери) - він розробив формулу зміни династії. Голова династій - це володар Неба, який втілює в собі якнайбільшу кількість чеснот "де", цими чеснотами вважалися: чесність, благодать, справедливість тощо. 

В стародавньому Китаї сформувалися до першого тисячоліття до нашої ери чотири основні політично-релігійні течій:
\begin{itemize}
\item Конфуціанство - було засноване Конфуцієм. Роки життя 551-479 до нашої ери;
\item Моїзм - засновник Мао Цзи. Основним для цих людей було знання, як певна сила. Знання ґрунтувалися на почуттях людини і саме вони мали бути основою держави;
\item Легізм - засновник Шан Ян. Цей погляд наближався до наукового знання;
\item Даосизм - це вже майже релігія. Засновник Лао Цзи.
\end{itemize}

Політичні ідеї стародавнього світу проголошували:
\begin{enumerate}
\item Політику, як результат божественного провидіння (але це все ще без християнства);
\item Верховні правителі мають бути виконавцями божественної волі на землі
\item Повноваження володаря священні;
\item Влада недоторкана та спадкоємна;
\item Соціальна нерівність визнана і раціональна.
\end{enumerate}
\subsection{Стародавня Греція}
\subsubsection{Перший етап: 9-6 століття до нашої ери}
Це етап становлення Грецької державності. Цю епоху назвали "Епоха семи мудреців". Це піфогорійська школа. В цей час відбувалося формування поглядів на державу.
\subsubsection{Другий етап: 5-перша половина 4 століття до нашої ери}
Школа софістів активно розвивалася в цей час.

Платон 427-347 роки до нашої ери.За Платоном жителі держави мали бути наділені трьома основними здібностями: розумовою, вольовою та афективною (тобто чуттєвою). Разом із цими здібностями Платон поділяв жителів на стани: філософи, воїни, прості люди, до цього стану додавалися раби. 

Арістотель 384-322 рік до нашої ери. Він був вихідцем зі знатної сім’ї. Також, він був вчителем Александра Македонського. За Арістотелем були визначені правильні та неправильні форми устрою держави. \textbf{Правильними} є \textit{монархія} (правління одного), \textit{аристократія}(правління небагатьох), \textit{політія} (правління багатьох людей, які можуть наблизитися до влади в наслідок своїх знань). \textbf{Неправильні} формами є переродженням правильних, тобто \textit{тиранія} (монарх переродився в тирана), \textit{олігархія} (правління матеріальних благ), \textit{демократія} (влада народу). 
\subsection{Римська імперія}
Третє століття до нашої ери. Вся проблема в людях. Концентрація держави на проблемах правового спілкування. Посилення правового захисту майнових відносин, за умови яких держава виступає як публічна правова спільнота. Про-республіканська форма правління  поступається про-монархічному режиму. І так до 17-18 століття.
\subsubsection{Загальні висновки}
\begin{itemize}
\item Відбувається перехід від міфологічного сприйняття світу до раціонально-логічного сприйняття;
\item Починається аналіз демократії як основи цивілізаційного розвитку;
\item Викладається концепція ідеальної держави, ідеального громадянина та ідеального правителя;
\item Держава є результатом історичного процесу;
\item Моральність має бути основою політики стосовно удосконалення особистості;
\item Напрацювання механізмів законотворчого і правозастосовчого процесів;
\item Розвиток системи рабовласницького права в Римській імперії, в якому відокремлюється природне право, приватне право та публічне право.
\end{itemize}
\subsection{Епоха Середньовіччя}
\indent Августин Блаженний (354-430 рр.). Основний твір (трактат) "О граде божем". Основна ідея твору: держава та правові установи в ній посилаються людям за їх гріхи.

Фома Аквінський (1226-1274 рр.) Основний науковий твір - "О правлении государев". Основна ідея полягає в тому, що влада не має протирічити інтересами церкви. Якщо таке відбувається, то піддані держави в праві цьому протистояти. Він благословляв відносини господства та підкорення самій сутності влади, тому що це продиктовано велінням Бога. Бог - сутність та інтелект, людина - потенція сутності.
\subsection{Епоха Відродження}
Епоха Відродження характеризується. Тісно повз’язана із виникненням у надрах відмираючого феодалізму паростків нових капіталістичних відносин. Цей злам був обумовлений появою та економічним зростанням "молодих" буржуа. По-друге, відбувався стрімкий розвиток міст. І по-третє, ріст ремесла, торгівлі, технічних винаходів.