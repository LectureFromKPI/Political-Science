\section{Еліти}
\subsection{План}
\begin{enumerate}
\item Сутність поняття еліти, сучасні теоріії еліт 
\item Структура, функції та види еліт; основні системи відбору
\item феномен політичного лідерства, природа, сутність, принципи та методи функціонування
\end{enumerate}
\textbf{Елітарний} - входить в політологічну науку у 19 столітті та означає найкращу групу людей, які володіють певними якостями, що виділяють їх з решти суспільства, та ознаками, що дозволяють формувати поведінку, погляди та інші суспільні думки членів соціуму.\\
\textbf{Політична еліта} - нечисельна, достатньо самостійна вища, відносно привілейована група людей, що більшою чи меншою мірою володію видатними психічними, соціальними та політичними якостями та безпосередньо бере участь у прийнятті та здійсненні рішень, що стосуються використанн
\begin{itemize}
\item Еліта - це особи, які отримали найвищий індекс в галузі своєї діяльності(саме таке тлумачення розглядав Парето)
\item Еліта - найбільш активні  в політичному відношенні індивіди, які мають чітку орієнтацію на владу
\item Люди, що мають високий рівень соціального статусу і впливають на політичний процес
\item Харизматичні особистості формують еліту
\end{itemize}
Гайтана Моска. Роки життя: 1854-1941. Він виділяв три основні форми існування аристократії
Форми аристократії: 
\begin{itemize}
\item військова
\item фінансова
\item церковна
\end{itemize}
Три способи оновлення правлячого класу:

\begin{itemize}
\item успадкування місця в еліти
\item вибори
\item кооптація - додаткова вибірність, коли той чи інший індивід може потрапити у еліту
\end{itemize}
У розвитку політичної еліти прослідковуються  дві основні тенденції:
\begin{itemize}
\item аристократична  - принцип закритості, основний шлях - спадкування місця у еліти, основна небезпека - еліта починає вимирати 
\item демократична тенденція - чекає підготовлених керівників - людей, що найчастіше зробили себе самі
\end{itemize}  
Новий дослідник: Вільфредо Парето(1848-1923). Він взагалі історію людства розглядів як історію циркуляції еліт.
Він виділяв два типи еліт:
\begin{itemize}
\item правлячу
\item неправлячу(контреліта) 
\end{itemize}
Леви  - та елітарна група, якій притаманний консерватизм у поглядах, силові методи управління, відкритість та рішучість. Вважав, що така еліта має існувати у рамках стабільної політичної системи.\\
Лиси - прихильники політичних комбінацій, маніпуляцій, хитрощів, схильні до демагогії, радикальних перетворень у суспільстві. Зауважував, що саме такі еліти з’являються у нестабільні періоди і вони вимагають прагматичних керівників.\\
Роберт Міхельс(1876-1936) - засновник "залізного закону олігархії". Пояснював, що будь-яке суспільство потребує елітарності, а влада еліт залежить від організованності. Сама організація суспільства здатна відтворювати еліту і будь-яка соціальна організація схильна до олігархізації.
!!ЗАЛЕЖНИК!!! матеріальних статків.\\
Функції політичної еліти:
\begin{itemize}
\item визначення стратегії та тактики, зовнішньої та внутрішньої політики держави
\item керівництво та управління суспільством
\item розробка та прийняття політичних рішень
\item мобілізація мас на досягнення певних політичних цілей
\item вираження інтересів та потреб різних соціальних груп
\item розробка ідеологій, ціннісних орієнтирів, ідей та ідеалів
\item забезпечення вертикальної комунікації між владою та масами, а також укріплення горизонтальних зв’язків в суспільстві 
\end{itemize}
Сучасна еліта поділяється на:
\begin{itemize}
\item управлінську - бюрократи
\item формальна еліта - виробляє загальну стратегію суспільного розвитку
\item відкрита еліта - еліта, здатна сприймати інновації та стабілізувати соціальне положення
\item закриті еліти - ті, що нездатні сприймати інновації, живуть за своїми консервативними правилами
\end{itemize}
Два основні шляхи формування еліти, що виділяє політична практика відповідно до двох систем відбору:
\begin{itemize}
\item система гільдій - об’єднання, корпорація ставилась в основу, яка при  відборі можливих  кандидатів робила акцент на їх політичних здібностях та перевагах, суворо діяла в рамках дотримання певних правил та настанов політичної організації. Характеризувалась більшим професіоналізмом, досвідом, передбачуваністю дій, проте схильна до бюрократизації, консерватизму, має небезпеку перетворення еліти в закриту касту
\item антрепренерська система - орієнтація на здібності особистості, які виділяють його серед інших креативністю мислення, здатністю переконувати, вмінням подобатись, адекватним прийняттям молодих лідерів, прийняттям та застосуванням інновацій. 
\end{itemize}
\textbf{Політичне лідерство} - процес взаємодії між людьми, під час якого наділені реальною владою авторитетні люди здійснюють легітимний вплив на суспільство чи на певну його частину, яка добровільно їм віддає частину своїх повноважень, що з точки зору політики є владними, а також передає їм частину своїх політичних прав.\\
Богомол - це летючий тарган(с) Заяць.\\
\begin{itemize}
\item вміння впливати на людей
\item завоювання симпатій
\item мобілізувати здатність адекватно впливати на членів суспільства, реалізуючи їх інтереси
\item мати авторитет
\item ораторські та організаційні здібності
\item популярність 
\item гострий розум
\item компетентність
\item рішучість  в прийнятті рішень
\item здатність висувати нові ідеї
\item брати на себе відповідальність, приймати рішення в екстермальних ситуаціях
\item досвід
\end{itemize}
Єже Вятр. Чотири основні типи еліт(духотомії):
\begin{itemize}
\item ставлення до власних можливостей(лідери-ідеологи та лідери-прагмати)
\item лідери по відношенню стосовно своїх прихильників(харизматики та лідери-представники)
\item стосовно супротивників(лідери-угодовці(схильні ідти з супротивниками на угоди) та лідер-фанатик, що прагне до загострення конфлікта)
\item за способом оцінювання дійсності - відкритий лідер та лідер-догматик
\end{itemize} 
