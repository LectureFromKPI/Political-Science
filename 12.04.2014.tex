\section{Характеристика політичної системи}
\begin{itemize}
\item Поняття політичної системи 
\item Структура та ф-ї політичної системи
\item Типологія політичної системи
\end{itemize}
В рад. часи був термін "політична організація суспільства". 
\textbf{Політична система} - одна з підсистем суспільства як соціальної системи, що забезпечує здатність суспільства реагувати на потреби індивіда та адаптуватись до умов свого функціонування, здійснювати вплив на всі сторони суспільного життя.\\
\textbf{Політична система суспільства} - інтегрована сукупність відносин влади, суб’єктів політики, держави, державних і недержавних інститутів, покликаних виконувати політичні функції щодо захисту, гармонізації інтересів, соціальних угрупувань, спільнот, соціальних груп, забезпечуючи таким чином стабільність і соціальний порядок щодо життєдіяльності суспільства.\\
\textbf{Риси політичної системи}
\begin{enumerate}
\item Цілісність
\item Структурність
\item Наявність підсистем
\item Функціональність
\item Певні стимули, здатність до самозбереження
\end{enumerate}
Вчений Істон визначив, що будь-якій системі потрібні вхід та  вихід, вхід забезпечують деякі вимоги та дії системи, на виході мають бути рішення, якщо вони адекватні, то система няшна.\\
Девід Істон дав визначення політичній системі як сукупності різноманітних видів діяльності, які впливають на прийняття та виконання певних рішень.(система включає вхід, вихід і конверсію - те, що усередині системи)\\
Політична система має складатися з трьох основних частин: 
\begin{enumerate}
\item культурні цінності
\item владні структури
\item поведінка політиків та пересічних членів суспільства
\end{enumerate}
Структура політичної системи
\begin{itemize}
\item завоювання та здійснення влади
\item політичні принципи, норми та метрики(формують політичну поведінку, свідомість людини, т. ч. відповідають за мету, поставлену політичною системою)
\item політична організація(система інститутів, в межах яких здійснюється політичне життя, наприклад держава, політ. партії, політ. рухи)
\item політична культура - сукупність знань, уявлень та цінностей, установок та стандартів, завдяки яким політичні суб’єкти ефективно виконують свої ролі 
\item політична свідомість - це стійкі орієнтації та установки людей, що стосуються політичної системи
\item засоби масової інформації (збирання, обробк та поширення інформції віддається на розсуд ЗМІ)
\end{itemize}
Підсистеми політичної системи:
\begin{enumerate}
\item інституціональна підсистема - комплекс формальних і неформальних політичних інститутів
\item функціональна підсистема(сукупність ролей, функцій, суб’єктів політики та політичних відносин)
\item нормативно-регулююча підсистема (та, що встановлює правила гри: норми, цінності, звичаї, традиції)
\item комунікативна підсистема, до якої входить сукупність форм взаємодії суб’єктів політики
\end{enumerate}
\textbf{Функції політичної системи}
\begin{itemize}
\item вироблення чіткого політичного курсу держави; визначення цілей, завдань та шляхів розвитку суспільства
\item розробка конкретних програм дій та організація їх виконання
\item мобілізація ресурсів для виконання завдань системи
\item інтеграція та об’єднання в єдине ціле всіх елементів структури 
\item забезпечення стійкого розвитку суспільства
\end{itemize}
Типологія політичної системи
\begin{enumerate}
\item за ступенем відкритості до зовнішнього середовища та здатністю сприймати інновації системи бувають 	      \begin{itemize}
            \item відкриті - динамічна, обмін ресурсами з іншими
            \item закриті - жорстка фіксація структури, орієнтація на власні цінності та обмежені у зв’язках із зовнішнім середовищем
            \end{itemize}
\item за характером політичного режиму
      \begin{itemize}
        \item тоталітарний
        \item авторитарний
        \item демократичний
       \end{itemize}
\item за типом формації 
   \begin{itemize}
   \item рабовласницька
   \item феодальна
   \item буржуазна
   \item соціалістична
   \end{itemize}
\item територіально-просторова
  \begin{itemize}
  \item континентальні
  \item острівні
  \end{itemize}
\item відповідність соціальній системі
  \begin{itemize}
  \item військові
  \item громадянські
  \item тощо
  \end{itemize}
\end{enumerate}
 Габріель Алмонд - перша половина 20 століття. Чотири типи політичних систем:
 \begin{enumerate}
 \item англо-американська політична система: США, Великобританія, Канада, Австралія. Риси: прагматизм, раціоналізм. Цінності: свобода особистості, безпека, власність, добробут, індивідуалізм, терпимість, толерантність. Система здатна до саморегуляції, стабільна, ефективна.
 \item Континентально-європейська політична система: Італія, Німеччина, Франція тощо.Риси: фрагментарність політичної культури(площа невелика, але країн багато, і кожна вважає себе самою розумною), співіснування нових та старих культур, політичні традиції, розділ політичних ролей в межах класу, групи, а не суспільства в цілому, багатоманітність політичних партій з різними ідеологіями
 \item Доіндустріальні політичні системи: країни Азії, Африки, Латинську Америку. Інтеграція суспільства досягається шляхом насилля, перехрещення різних політичних культур, поширення авторитаризму та влади однієї партії, відсутність розподілу владних повноважень, обмеження прав та свобод
 \item Тоталітарна політична система. Риси: пріоритет расових, класових, релігійних, національних цінностей, контроль за усіма сферами суспільного життя, концентрація влади в руках бюрократичного, монополія правлячої партії, заідеологізованість, розширення функцій репресивних органів, функціональна нестабільність: часи фашистської Італії, Радянський Союз.
 \item  
 \end{enumerate}  
