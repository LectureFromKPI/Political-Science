\section{Політична влада та її субєкти в Україні} 
\subsection{Питання семінару}
\begin{enumerate}
\item Сутність політики як соціального явища.
\item Форми, рівні та функції політики.
\item Державна політика та її функції.
\item Сутність влади як соціального явища.
\item Політична влада та її суб’єкти.
\end{enumerate}
\subsection{Сутність політики як соціального явища}
У питанні визначення поняття політика в суспільствознавстві Заходу сформувались дві основні позиції:

\textbf{Одна з позицій} - традиційна. Політика визначається через суть і зміст діяльності держави, через участь людей в здійсненні або опануванні державної влади. Політика - наука про державу більш давня і водночас більш близька до здорового глузду. Політика як наука бере початок з Арістотеля, для якого політика є вивчення управління містом (полісом), державою. Розвиток національних держав посилило таку позицію у визначенні суті політики, як соціального явища і науки. Політика - пізнання всього, що має стосуватися мистецтва управляти державою і вести стосунки, зв'язки з іншими державами. Найпоширеніша на Заході концепція політичної соціології визначає політику як науку про владу, про управління, про авторитет, про командування в усіх людських спільностях і соціальних групах, а не тільки у національному суспільстві (Макс Вебер, Моріс Дювер-же, Гарольд Лассуел, Роберт Даль та ін.). Політика створює особливу сферу суспільного життя, яку точніше іменувати державно-владною і реалізується в ній. Вплив політики на економіку, культуру і вплив економіки на політику та ін. розглядається як взаємодія різних сфер суспільного життя.

\textbf{Друга позиція}. Політика визначається як певний вид соціальної діяльності, не обов'язково зв'язаної з державною владою Фяц, політологів Заходу вважають, що політика є спосіб з'ясування і упорядкування суспільних справ, що особливо стосуються розподілу дефіцитних ресурсів, принципів, за якими здійснюється розподіл ресурсів, засобів, завдяки яким люди або соціальні спільності мають і тримають контроль над ситуацією. Влада і держава, в межах другої позиції погляду на політику, не розглядаються апріорно як щось відмінне від влади та інших соціальних спільностей людей, якщо відмінність і є то порівняне вивчення влади в усіх соціальних спільностях людей, дає можливість їх виявляти. Це означає, що політика є, насамперед, суспільна діяльність, звернена на соціальні і матеріальні взаємовідносини людей, одержуючи різноманітні відображення у різних сферах і постійно змінюються. Аналізуючи політику як соціальне явище, варто мати на увазі, що в суспільстві з складною соціальною структурою і за наявності публічної влади - держави, важко знайти явища і процеси абсолютно вільні від політичних відтінків, оскільки більшість з них зв'язано з інтересами тих або інших соціальних спільностей, верств, класів, що борються за завоювання або утримання влади. Інша справа, міра «політичності» є різною. Найсильніша міра «політичності» в сфері законодавчій і виконавчій діяльності органів державної влади, в органах, що забезпечують національну безпеку, а також в діяльності політичних партій, суспільно-політичних рухів, масових громадських і суспільних об'єднань та деяких професійних спілок творчих добровільних товариств.

Отже, визначення поняття і суті політики дають можливість зробити висновок, що політика не створює особливу сферу суспільного життя, але й не заперечує наявності власної державно-владної сфери, інтегрує різні сторони явищ, але й не зводиться до жодного з них.

Функціонування політики в суспільстві, її суспільне буття визначається суспільством, державою, владою і змістом, суттю самої політики, універсальними і незмінними властивостями або принципами її існування, що склалися історично. їх сукупність і взаємні відносини становлять основу суспільного буття політики. Суспільне буття політики визначається двояко: загальними властивостями політики, що склалися історично і конкретними умовами суспільного життя і розвитку, реальними властивостями самої політики, суспільствами і особливостями їх розвитку.
\subsection{Форми, рівні та функції політики}
\textbf{Політика} - одна з найважливіших сфер життєдіяльності суспільства, взаємин різних соціальних груп та індивідів щодо утримання й реалізації влади задля здійснення своїх суспільно значущих інтересів і потреб, вироблення обов’язкових для всього суспільства рішень.

\noindent\textbf{В політології розрізняють три рівні існування політики:}

\textbf{Мегарівень} охоплює світовий політичний процес у всіх його взаємозв’язках і опосередкуваннях, весь геополітичний простір, міждержавні відносини і міжнародну політичну проблематику. На цьому рівні аналізується діяльність міжнародних суб’єктів політики та використовуються загальнофілософські методологічні підходи.

\textbf{Макрорівень} характеризує функціонування політики в національно-державному та регіональному масштабі. Об’єктом аналізу стають центральна і місцева влада. На цьому рівні застосовуються структурно-функціональний, системний, порівняльний, комунікаційний та інші методи дослідження.

\textbf{Мікрорівень} охоплює окремі організації-партії, профспілки, корпорації тощо. Це такий рівень аналізу політики, що є максимально наближеним до місцевого матеріалу, дозволяє проводити конкретні соціологічні дослідження і процедури: узагальнення статистичних даних, анкетування, контент-аналіз політичних документів (партійних програм, правових актів тощо).

\noindent\textbf{Функціонування політики розмежовують за різними критеріями:}
\begin{itemize}
\item за сферами суспільного життя (економічна, соціальна, культурна, національна, військова тощо);
\item за орієнтацією (внутрішня, зовнішня);
\item за масштабами (міжнародна, світова, локальна, регіональна);
\item за носіями й суб`єктами (політика держави, партії, руху, особи);
\item за терміном дії, (коротко-, середньо-, довгострокова).
\end{itemize}
\noindent\textbf{Політика має складну структуру. Найчастіше виокремлюють у ній три головні елементи:}
\begin{enumerate}
\item Політичні відносини та політична діяльність (відображають стійкий характер взаємодії суспільних груп між собою та з інститутами влади).
\item Політична свідомість (свідчить про принципову залежність політичного життя від свідомого ставлення людей до своїх владно значущих інтересів).
\item Політична організація (характеризує роль інститутів публічної влади як центрів управління й регулювання суспільних процесів). Охоплює такі елементи: сукупність органів законодавчої, виконавчої й судової гілок влади; партійні та громадсько-політичні інститути; групи тиску; громадські організації та рухи тощо.
\end{enumerate}
\noindent\textbf{У політології виокремлюють (здебільшого на загальнодержавному рівні) такі функції політики:}
\begin{itemize}
\item задоволення владно значущих інтересів усіх груп і верств суспільства;
\item раціоналізація конфліктів і протиріч, спрямування їх у русло цивілізованого діалогу громадян і держави;
\item примус в інтересах окремих верств населення або суспільства загалом;
\item інтеграція різних верств населення шляхом підпорядкування їхніх інтересів інтересам усього суспільства;
\item соціалізація особистості (залучення її до складного світу суспільних відносин);
\item забезпечення послідовності та інноваційності (оновлюваності) соціального розвитку як суспільства в цілому, так і окремої людини.
\end{itemize}
Функції політики засвідчують її всеосяжний характер, неперервний вплив на суспільство й неперехідне значення для врегулювання суспільних відносин. Політика тісно пов`язана з різними сферами суспільного життя: економікою, мораллю, правом, релігією, культурою, екологією тощо.

Сучасною політичною наукою та всією громадсько-політичною думкою сформовано чіткі \textbf{засади демократичної політики}:
\begin{itemize}
\item оптимальне поєднання класового й загальнолюдського, універсального й національного;
\item гуманістична спрямованість, подолання технократизму, насильства, злочинності;
\item демократизм і моральність у здійсненні політики;
\item громадянськість і патріотизм.
\end{itemize}

\noindent\textbf{Під час вироблення та реалізації політики важливо враховувати такі основні чинники:}
\begin{itemize}
\item конкретно історичні умови розвитку соціуму, геополітичні умови й географічне розташування держави;
\item рівень участі чи відчуження населення щодо влади й державно-суспільних справ;
\item спрямованість національної ментальності, рівень розвитку політичної та правової свідомості;
\item етнонаціональний і демографічний чинники суспільного розвитку;
\item відповідність політичних ідеалів і завдань історичній традиції, політичним цінностям певного суспільства, а також принципам гуманізму й соціальної справедливості;
\item реальна міжнародна ситуація і ставлення до держави світової громадськості.
\end{itemize}
\subsection{Державна політика та її функції}
Сукупність цілей, завдань і функцій держави, що нею практично реалізуються, засобів і методів, які при цьому використовуються, віддзеркалюють сутність держави. Їх формування, правове закріплення здійснюється політичною системою та виражається \textbf{державною політикою}.

Державна політика являє собою оптимальний синтез об’єктивних тенденцій суспільного розвитку і переважну більшість суб’єктивних тверджень людей про свої інтереси в ньому.

Інтегровано суспільні інтереси, певні правила їх реалізації викладені в \textbf{Конституції України}. Фактично вони характеризують основні ознаки держави, визначають суспільний устрій, являють собою цілі держави тощо.

\textbf{Саме держава є визначальним суб’єктом}. Тільки вона може здійснювати заходи, підпорядковані стратегічним національним інтересам, мета яких — цілеспрямоване формування громадянського суспільства з розвинутою економічною системою, підтримка розвитку його складових та нівелювання структурних диспропорцій, що неодмінно виникають у процесах формування та розвитку. Тим самим держава ототожнюється із суспільством, оскільки власне існування держави зумовлене стабільністю та гармонійним розвитком суспільства.

Державна політика повинна закріплюватися державно-правовими актами, бути відомою і зрозумілою суспільству. Необхідно зауважити, що всі політичні проблеми врешті-решт мають інституційний характер, оскільки в державній політиці важливе юридичне оформлення політичних проблем, і що прогрес на шляху до реалізації державної політики можливий за інституційного контролю над владою.

Будь-яка державна політика реалізується в певних умовах з використанням адекватних їй засобів. Формулюючи сутність державної політики, актуально зазначити умови і засоби, які їй сприяють і можуть бути практично використані.

Серед \textbf{умов} доцільно виділити такі:
\begin{itemize}
\item державно-правові, які полягають у створенні узгодженого правового простору країни, певною мірою ідентичного, структурованого, який дозволяє максимально використовувати технології політичної, економічної, соціальної та іншої діяльності зі своїми особливостями та спеціалізаціями;
\item соціально-психологічні, які включають у себе усвідомлення нових життєвих орієнтирів, відхід від ілюзій, від усього того, що не відповідає реаліям життя і не народжує розбудовчу енергію людей;
\item діяльнісно-практичні, коли рішення, дії, операції, процедури, вчинки тощо спрямовані на досягнення цілей державної політики та здійснюються в її руслі, «просувають» цю політику і наочно розкривають її цінність для суспільства.
\end{itemize}
У формуванні державної політики важливим елементом є \textbf{визначення загального інтересу та волі більшості громадян}. Воно здійснюється:
\begin{itemize}
\item волевиявленням народу під час проведення виборів Президента України, народних депутатів і депутатів органів місцевого самоврядування, референдумів;
\item волевиявленням суб’єктів політичної системи, особливо під час формування фракцій у Верховній Раді України, виборчих блоків, об’єднань і рухів у період виборчих компаній;
\item впливом професійних спілок, об’єднань суб’єктів господарювання, професійних і соціальних угруповань тощо.
\end{itemize}
\textbf{Засоби реалізації державної політики} досить різні: від запровадження та ефективного використання різноманітних форм власності, ринкової економіки до удосконалення систем інформації, освіти і виховання тощо. Політика безпосередньо та діалектично пов’язана з мораллю, духовною, релігійною та іншими сферами суспільного життя.

Розглянемо \textbf{процес формування державної політики}. Її засади формуються на підставі конституційних положень, щорічних та позачергових послань Президента України про внутрішнє та зовнішнє становище України, актів Верховної Ради України та довгострокових програм розвитку. Саме в них на підставі аналізу економічної, соціальної та політичної ситуації на засадах дотримання основних цілей і ознак держави формуються пріоритетні напрями економічного та соціального розвитку держави.

Підсумовуючи, можна сказати, що цільові настанови в економічній діяльності держави спрямовуються на запровадження принципів результативності, ефективності, рівності між суб’єктами господарювання і стабільності економічного та соціального розвитку суспільства. Реалізація цих принципів передбачає створення умов для виробництва зростаючої кількості товарів і послуг на технологічній основі, що постійно удосконалюється; мінімізацію витрат в умовах використання обмежених матеріальних і виробничих ресурсів; зміцнення позицій держави на світовому ринку; створення робочих місць для тих, хто бажає та може працювати. Це передбачає і економічну свободу для всіх видів господарської діяльності, споживачів і продавців на ринку, щоб вони мали свободу вибору.
\subsection{Сутність влади як соціального явища}
Будь-яка система має системоутворюючий компонент. Для політичної системи ним є політична влада. Вона інтегрує всі елементи системи, навколо неї точиться політична боротьба, вона – джерело соціального управління, яке, в свою чергу, є засобом здійснення влади. Отже, влада є необхідним регулятором життєдіяльності суспільства, його розвитку та єдності.
\noindent\textbf{Існують різні підходи до пояснення причин владних відносин:}
\textit{Біологічний підхід} – визнає владу притаманною біологічній природі людини. А оскільки біологічна природа людини і тварин є спільною, то визнається наявність владних відносин не тільки в суспільстві, а й у тваринному світі. Арістотель розглядав владу в суспільстві як продовження влади в природі. Однак, відносини в суспільстві мають свідомий характер, тоді як у тваринному світі вони зумовлюються інстинктами і рефлексами.

\textit{Психологічний підхід} – розглядає владу крізь призму міжособистісних стосунків, де більш сильна особистість домінує над слабшими, нав’язуючи останнім свою волю. Особливу увагу прихильники психологічного підходу приділяють аналізу поведінки суб’єктів політичних відносин.

\textit{Антропологічний підхід} (від грецьк. anthropos – людина) – пов’язує поняття політичної влади, а отже, й політики з суспільною природою людини і поширює його на всі соціальні, в тому числі й докласові, утворення. Прихильники цього підходу доводять наявність політичної влади на всіх етапах розвитку суспільства.

\textit{Політологічний підхід} – на відміну від антропологічного ґрунтується на органічному зв’язку влади й політики, пов’язує їх існування лише з певними етапами суспільного розвитку, для яких характерною є наявність спеціальних суспільних інститутів здійснення влади, насамперед держави.

\textbf{Оскільки влада є суто суспільним відношенням}, у якому задіяні наділені свідомістю і волею люди, а переважний вплив одних людей на інших є вольовим відношенням між ними, то \textit{влада може бути визначена як вольове відношення між людьми, тобто таке відношення, за якого одні люди можуть нав’язувати свою волю іншим}.

Попри розмаїття підходів до \textbf{означення влади як соціального явища} більшість сучасних дослідників характеризують владу як одне з фундаментальних начал суспільства, його першооснову. Влада і владні відносини наявні скрізь, де є соціальні об`єднання людей, у всіх сферах суспільного життя — економічній, політичній, духовній, сімейній тощо. Виділяють і різні рівні вияву влади. Так, на рівні суспільства в цілому влада охоплює найскладніші соціальні й політичні відносини: влада на публічному (асоціативному) рівні регламентує діяльність колективів і відносини в них (громадські організації, спілки, виробничі колективи); влада є важливим атрибутом соціальних відносин і для рівня особистісного, приватного життя, функціонування малих соціальних груп.

\textbf{У суспільному житті} влада обумовлюється необхідністю регулювати різноманітні й суперечливі соціальні інтереси суб`єктів, узгоджувати й регламентувати взаємовідносини, починаючи від індивідуальних, групових і аж до загальнодержавних.

Отже, в основі влади і владних відносин — соціальні інтереси громадян, певних страт, соціальних класів. \textbf{Соціальні інтереси} — це об`єктивно зумовлені мотиви діяльності соціальних суб`єктів, які складаються з усвідомлення власних потреб і з`ясування умов та засобів задоволення їх. Прагнення реалізувати соціальні інтереси спонукає учасників використовувати різноманітні засоби і ресурси, створювати відповідні умови життєдіяльності та соціальні відносини.
\subsection{Політична влада та її суб’єкти}
\textbf{Політична влада} – здатність і можливість здійснювати визначальний вплив на діяльність, поведінку людей та їх об’єднань за допомогою волі, авторитету, права, насильства.

Поняття „\textbf{політична влада}” ширше від поняття „\textbf{державна влада}”. По-перше, політична влада виникла раніше від державної, ще в додержавну добу. По-друге, не кожна політична влада є владою державною (наприклад, влада партій, рухів, громадських організацій), хоча будь-яка державна влада – завжди політична. По-третє, державна влада специфічна: тільки вона володіє монополією на примус, правом видавати закони тощо.

\noindent\textbf{Функції політичної влади:}
\begin{itemize}
\item інтегративна (полягає в об'єднанні соціально-політичних сил суспільства);
\item регулятивна (спрямовує політичну волю мас на регулювання життєдіяльності суспільства, правотворчість);
\item мотиваційна (формування мотивів політичної діяльності, передусім загальнозначущих);
\item стабілізуюча (націленість на стійкий розвиток політичної системи, громадянського суспільства).
\end{itemize}
Основними рисами політичної влади є легальність, легітимність, верховенство, вплив, ефективність і результативність.

\textbf{Суб’єкт політичної влади}. Існує думка, що поняття „\textbf{суб”єкт влади}” і „\textbf{носій влади}” нетотожні. \textbf{Суб’єкт влади} – це соціальні групи, насамперед панівні класи, політичні еліти, окремі лідери; \textbf{носії влади	} – державні та інші політичні організації, органи і установи, утворені для реалізації інтересів політично домінуючих соціальних груп. Існує й інша класифікація владних суб’єктів. Згідно з нею суб’єкти влади умовно поділяють на первинні й вторинні:
\begin{description}
\item[Первиним суб’єктом] а республіканського, демократичного правління є \textit{народ} – носій суверенітету і єдине джерело влади в державі. Він здійснює владу безпосередньо і через органи державної влади та місцевого самоврядування. Поняття народ неоднорідне: основними суб’єктами влади є великі групи населення, об’єднані спільністю корінних інтересів і цілей; неосновними – невеликі етнічні групи, релігійні громади тощо.
\item[Вторинні суб’єкти носіїв влади] - малі групи, представницькі колективи, партії, асоційовані групи, групи партикулярних (приватних, неофіційних) інтересів тощо. \textit{Суверенним суб’єктом} політичної влади є громадянин держави, наділений конституційними правами та обов’язками. Суттєву роль у владних відносинах відіграють політичні лідери. Сукупним (колективним) носієм політичної влади є сама політична система суспільства як спосіб організації і розвитку соціальних спільнот і їх стосунків.
\end{description}