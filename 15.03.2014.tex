\section{Влада як соціально-політичний феномен} 
\subsection{Питання лекції}
\begin{enumerate}
\item Характеристика поняття "влада". Політична влада. Суб’єкт політичних відносин;
\item Основні функції, структура та класифікація влади.
\end{enumerate}
Влада виступає як певне соціальне явище і її структура визначається трьома позиціями:
\begin{enumerate}
\item Пануючий інтерес;
\item Воля, яка його регламентує (закони, політичні рішення тощо);
\item Засоби забезпечення інтересу (моральні та матеріальні засоби).
\end{enumerate}
Владні відносини:
\begin{itemize}
\item Наявність партнерів;
\item Волевиявлення владерюючого стосовно підлеглого;
\item Обов’язкове підкорення тому, хто владерю;
\item Наявність соціальних норм, що закріплюють право одних видавати акти, накази, а іншим підкорюватись ним.
\end{itemize}

\textbf{Політична влада} - це можливість і здатність суб’єктів політики впливати на політичну поведінку учасників політичних відносин, на прийняття та реалізацію політичних рішень. 

\textbf{Політична влада} - це здатність індивіда, соціальної спільноти до вияву своєї волі у політиці на основі осмисленого політичного інтересу та сформованих політичних потреб.

\subsection{Властивості влади}
\begin{itemize}
\item \textbf{Суверенітет} - це необмеженість та неподільність влади в рамках певної території. Суверенітет втрачається тоді, коли зовнішній вплив стає сильніше за внутрішні важелі;
\item Відповідальність перед народом;
\item Вольова спрямованість влади через закон, право, певні нормативи;
\item Примусовість влади;
\item \textbf{Легітимність влади} - це здатність системи викликати віру народу в те, що її політичні інститути і лідери відповідають інтересам і моралі народу, підтримують її та існують в рамках сформованого законодавчого поля.
\end{itemize}
\subsection{Легітимність влади}
Три типи легітимності влади за Максом Вебером:
\begin{itemize}
\item \textbf{Традиційний} - заснований на звичці підкорятися, вірити в непохитність і законність існуючих порядків;
\item \textbf{Харизматичний} тип, що базується на вірі у виняткові якості керівника, якій наділений винятковими здібностями та мистецтвом керувати;
\item \textbf{Раціонально-правовий}, що ґрунтується на загальному визнанні правомірності системи законів в рамках існуючої владної системи, де взаємовідносини базуються на схемі "демократія=довіра".
\end{itemize}
\subsection{Структура влади}
Структура влади:
\begin{itemize}
\item Суб’єкт та об’єкт влади;
\item Джерела влади;
\item Ресурси влади;
\item Функції влади.
\end{itemize}

\textbf{Суб’єктами} влади є особистість, організація, суспільна група, яка здатна творити політику чи відносно самостійно і постійно брати участь у політичному житті відповідно до своїх інтересів, впливати на становище інших, мати можливість викликати суттєві зміни в політичних відносинах.\\
Виділяють такі суб’єкти політики:
\begin{itemize}
\item класи;
\item еліти;
\item зацікавлені групи;
\item тощо.
\end{itemize}
\subsection{Основні шляхи досягнення влади}
\begin{itemize}
\item політична революція;
\item контрреволюція;
\item політичні реформи;
\item політичний переворот;
\item спадковість;
\item вибори.
\end{itemize}
\subsection{Засоби здійснення влади}
\begin{itemize}
\item право;
\item авторитет;
\item переконання;
\item традиції;
\item маніпулювання;
\item насильство.
\end{itemize}
\subsection{Джерела влади}
\begin{itemize}
\item сила;
\item багатство;
\item становище в суспільстві;
\item знання;
\item інформація або інформатизованість;
\item належність до певної організації.
\end{itemize}
\subsection{Форми прояву влади}
\begin{itemize}
\item примус;
\item переконання;
\item сила;
\item маніпуляція;
\item авторитет;
\item заохочення;
\item контроль;
\item управління;
\item покарання.
\end{itemize}
\subsection{Класифікація влади}
\begin{itemize}
\item Галузь функціонування:
	\begin{itemize}
	\item законодавча влада;
	\item виконавча влада;
	\item судова влада;
	\item світська влада;
	\item духовна влада;
	\item сімейна влада;
	\item економічна влада.
	\end{itemize}
\item За суб’єктами влади:
	\begin{itemize}
	\item \textbf{Первинних суб’єктів влади} - це масштабніші спільноти та організації, які мають можливість формувати партію та здійснювати владу;
	\item \textbf{Вторинних суб’єктів влади} - це можуть бути певні структури та особистості.
	\end{itemize}
\item Методи здійснення влади:
	\begin{itemize}
	\item панування;
	\item насилля;
	\item примус;
	\item переконання.
	\end{itemize}
\item За об’єктами влади:
	\begin{itemize}
	\item одноосібну;
	\item групову;
	\item колегіальну.
	\end{itemize}
\item Режим правління:
	\begin{itemize}
	\item тоталітарну;
	\item авторитарну;
	\item ліберально-демократичну;
	\item демократичну;
	\end{itemize}
\end{itemize}
\subsection{Ресурси влади}
\begin{itemize}
\item \textbf{Влада примусу} - покарання за небажання діяти або блокувати певний мотив;
\item Влада зв’язків;
\item Влада експертна;
\item \textbf{Нормативна влада} - яка ґрунтується на контролі виконання правил поведінки;
\item \textbf{Еталонна влада} - .
\end{itemize}