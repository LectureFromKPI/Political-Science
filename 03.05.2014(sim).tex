\section{Сутність, функції та механізми політичного лідерства, умови його формування в сучасній  Україні} 
\subsection{План}
\begin{enumerate}
\item Політичне лідерство: теорії, структура, основні функції.
\item Поняття та сутність політичного лідерства в Україні.
\item Проблеми формування  політичного лідерства в сучасній Україні.
\end{enumerate}
\subsection{Політичне лідерство: теорії, структура, основні функції}
\textbf{Політичне лідерство} в системі владних відносин займає особливе місце. У лідерстві найбільше яскраво проявляється «видимість» влади, її наочність. Політичні лідери персоніфікують собою владу. Вони мають такий величезний вплив, що він не зрівнюється із впливом інших суб'єктів політики.

Політичний лідер є символом певної спільноти, це особа, здатна реалізувати інтереси спільноти за допомогою влади, що дається йому цією спільнотою, політичних інтересів, організовувати та спрямовувати їхню активність, згуртувати довкола себе групу прихильників та повести їх за собою.

\textbf{Термін "лідер" походить від англ. leader} — ведучий, керівник, провідник, глава тощо. У сучасній політичній науці є різноманітні визначення поняття "політичне лідерство", зокрема:
\begin{itemize}
\item це влада, яку здійснюють один чи кілька індивідів з метою пробудження членів суспільства, нації до дій;
\item це стосунки між людьми у процесі спільної діяльності, за якої одна сторона забезпечує перевагу своєї волі над іншими;
\item це пос	тійний легітимний вплив владних осіб на суспільство, організацію чи групу.
При цьому прийнято виділяти такі ознаки політичного лідерства:
\item лідерство передбачає постійний вплив на навколишніх;
\item політичний вплив має бути всезагальним і стосуватися всіх членів керованої спільноти;
\item лідерство закріплюють певні норми, правила, привілеї, повноваження.
\end{itemize}
\textbf{Політичне лідерство має формальний і неформальний аспекти}. Формальний аспект лідерства пов'язаний із керівним місцем людини у суспільній ієрархії, з високою посадою, статусом, владою. Неформальний аспект лідерства полягає в особистих вроджених якостях людини, її здатності виконувати роль лідера.

\textbf{Проблема політичного лідерства має давню історію}. Ще в епоху античності лідером вважали особу, здатну творити історію. Певна суспільна ситуація вимагала свого лідера, вождя. Кожного історичного періоду виникали теорії, які визначали певний тип, образ та завдання лідера.

В епоху Відродження свою теорію політичного лідера створив італієць \textbf{Н. Макіавеллі}. Він вважав, що всі люди різні, але в масі своїй мають однакові звички і більше схильні до поганого, ніж до доброго. В основі людської природи — інтерес, або жадоба влади й наживи. Політичний лідер — це володар, який використовує будь-які засоби для наведення громадського ладу й збереження свого панування. Щодо рис лідера, то йому потрібно вдаватися до великих, віртуозних шахрайств, зради, які вимагають мужності, особистого впливу й авторитету.

Свою теорію надлюдини запропонував німецький мислитель \textbf{Ф. Ніцше} (1844—1900 рр.). Лідер, за його концепцією, — вищий біологічний тип людини, який ігнорує установлені мораль, культуру, політичні цінності. Сучасників Ніцше вважав утраченим поколінням, його герої — це герої майбутнього. Треба допомогти собі позбутися повсякденності, бути вищим за неї, щоб стати особою, здатною володіти й керувати. Це своєрідна концепція самовиховання, знищення в собі раба. Як спрощений, вульгаризований підхід до концепції, Ніцше застосував фашизм, що призвело до спотвореного сприйняття філософії німецького вченого, наклало на неї тавро людиноненависницької теорії.

\textbf{К. Маркс} (1848—1883 pp.) визначав лідера як особу, що має ряд певних особистих якостей (уміння, знання, авторитет, організаторський талант) та виражає інтереси й волю певного класу, зокрема пролетаріату. За марксистською концепцією лідерства, особистість може відіграти певну позитивну роль в історичному процесі за підтримки класів і широких верств населення, а також за сприятливих політичних умов.
\subsubsection{Існує низка концепцій, які обґрунтовують природу політичного лідерства:}
\begin{enumerate}
\item Теорія рис — пояснює феномен політичного лідерства наявністю видатних рис у людини, а саме: розуму, компетентності, організаційних здібностей тощо (Е. Богардус).
\item Ситуативна теорія — трактує лідера як продукт ситуації. Причина лідерства полягає не в індивідові та притаманних йому рисах, а в ролі, яку лідер має виконувати за конкретної ситуації (Ф. Фідлер).
\item Концепція послідовників — розкриває лідерство через взаємовідносини між лідером та його послідовниками, через вплив останніх на політичного лідера.
\item Психологічна концепція — в основі лідерства — прагнення людини перебороти певні комплекси і табу, досягти більшого, ніж вона має або може. Ця риса є в творчості, мистецтві та політиці.
\end{enumerate}
\textbf{У політологічному аспекті політичне лідерство} — це суспільно-політичний процес, за якого одна, а іноді кілька осіб беруть на себе і виконують роль глави, керівника, провідника певної соціальної групи, громадсько-політичної організації чи руху, держави або суспільства загалом.
\subsubsection{Залежно від стилю керівництва й політичної системи, де діє лідер, вирізняють типи:}
\begin{itemize}
\item диктаторський, за якого лідер прагне досягти мети, спираючись на страх покарання;
\item демократичний тип лідера, який підтримує дух співпраці, співучасті у вирішенні питань;
\item автократичний, який повинен мати високі професійні й особисті якості, аби перемогти опонентів;
\item плутократичний (часто це лідери "тіньової" економіки).
\end{itemize}
Відповідно до способу набуття та реалізації, легітимації влади, М. Вебер поділив політичних лідерів на такі основні типи:
\begin{itemize}
\item традиційні (вожді племен, монархи тощо), авторитет яких базується на традиціях, звичаях, часто освячується релігією;
\item  раціонально-легальні, які стають керівниками через загальноприйняті в суспільстві шляхи, внаслідок наполегливої праці, яка дозволила їм завоювати довіру електорату, в процесі якої вони довели здатність управляти. Такий тип лідерства характерний для спокійних періодів розвитку суспільства, коли основною цінністю для нього стає стабілізація політичної системи, використання закладених у ній можливостей;
\item харизматичні, що наділені, на думку мас, особливими надприродними здібностями, покликані до політичного життя вищими силами. Оптимальний момент для появи харизматичного лідера — ситуація глибокої системної соціальної кризи. У цей час суспільство, яке перебуває в стані аномії, переживає розчарування не лише в організаційній структурі політичного процесу, ідеології, яка забезпечує її функціонування, а й у лідерах, котрі її уособлюють. Воно готове до радикальних змін і прагне лідерів, що запропонують йому нові ідеї, покажуть нові* перспективи розвитку, поставлять якісно інші цілі. Суспільство в такі миті прагне чину, а не раціонального осмислення шляхів розвитку. Знищивши старих ідолів, воно знаходить для себе нових. Ними і стають лідери харизматичного типу — "сильні особистості без страху і сумнівів", які не лише абсолютно переконані в правоті своїх ідей, а й здатні переконати маси, повести їх за собою.
\end{itemize}
В основі першого типу лідерства — звичка, другий має раціональні корені, а останній базується переважно на емоціях. Лідери всіх типів мають як сильні, так і слабкі сторони, що необхідно враховувати громадянам при їх виборі, спілкуванні з ними. Класифікація М. Вебера є дещо спрощеною. Насправді у кожного політичного лідера поєднуються елементи всіх трьох типів, хоча один із них зазвичай переважає. У сучасній політології, залежно від завдань, які висувають дослідники, та обраних ними критеріїв оцінки, використовують й інші класифікації лідерства. Критерії класифікації політичних лідерів подано у таблиці.
\begin{tabular}{|c|c|}
\hline
Критерії класифікації & Політичні лідери\\
\hline
\begin{tabular}{c}За способом утвердження \\лідерства у групах і організаціях \end{tabular} & формальні неформальні\\
\hline
\begin{tabular}{c}За способом легітимізації\\ влади лідера у суспільстві  \end{tabular}& традиційні раціонально-легальні \\
\hline
\begin{tabular}{c}За стилем керівництва\\ та управління \end{tabular}&  \begin{tabular}{c}харизматичні,ліберальні,\\авторитарні, демократичні\end{tabular}\\
\hline
\begin{tabular}{c}За іміджем і ролевим \\ призначенням лідера \end{tabular}& \begin{tabular}{c}лідер-прапороносець,лідер-слуга,\\ лідер-торговець,лідер-пожежник,\\лідер-актор (демагог)\end{tabular}\\
\hline
\begin{tabular}{c}За стилем політичної поведінки \end{tabular}& \begin{tabular}{c}параноїдальний,демонстративний,\\депресивний,шизоїдний\end{tabular}\\
\hline
\begin{tabular}{c}За ставленням до політичної системи \end{tabular}& \begin{tabular}{c}функціональний,дисфункціональний,\\ нонконформістський,конформістський\end{tabular}\\
\hline
За масштабами лідерства & \begin{tabular}{c}загальнонаціональний,певного класу,\\соціальних груп, верств\end{tabular}\\
\hline
\end{tabular}
\subsubsection{Лідери всіх типів, виходячи із об'єктивних завдань, які перед ними ставить суспільство, мають виконувати такі функції:}
\begin{description}
\item[Вираження інтересів:] виявлення та формування політичних інтересів своїх прихильників, їх репрезентація. Варто зауважити, що видатні політичні лідери здатні не лише виявити дійсні інтереси груп, які вони репрезентують, позбавити їх ілюзій, уявних інтересів, а й суттєво змінювати систему політичних інтересів своїх прихильників, подаючи їм нові ідеї, знаходячи нетрадиційні шляхи їх реалізації. Проте захоплення харизматичних лідерів власними ідеями, їх прагнення впливати на формування системи інтересів первинних суб'єктів політики може мати й трагічні наслідки для суспільства
\item[Інтегративна функція] політичного лідерства полягає в тому, що на основі запропонованої лідером програми відбувається інтеграція дій його конституентів. В ідеалі програма лідера має передбачати задоволення інтересів і потреб кожної групи населення тієї чи іншої території. Хоча на практиці це неможливо через суперечність в інтересах, лідер мусить прагнути максимально узгодити всі інтереси й таким чином залучити на свій бік якомога ширші верстви населення. Інтегративна функція спрямована на підтримку цілісності і стабільності суспільства, громадянського миру і злагоди. Підтримка соціальної цілісності суспільства неможлива без цілеспрямованих зусиль щодо згуртування всіх соціальних спільностей. Подолання кризових явищ і своєчасне розв`язання суперечностей сприяють розвитку інтегративних суспільних процесів і підтримці цілісності соціальної системи. Лідери, які відстоюють вузько групові або лише суто класові інтереси, діють на шкоду суспільству, сприяють його розколу, розпалюють соціальні конфлікти.
\item[Організаторська, або прагматична, функція] політичного лідерства полягає у втіленні цілей і завдань, які стоять перед суспільством і відображені у програмі лідера, в конкретні дії. Йдеться про мобілізацію народних мас на втілення політичних програм і рішень у життя. Щоб організовувати і спрямовувати дії мас, політичний лідер повинен мати організаторські здібності, вміти завойовувати довіру мас, вести їх за собою. Невід`ємними складовими організаторської діяльності політичних лідерів є регулювання ходу суспільних перетворень та контроль за їх здійсненням.
\item[Комунікативна функція] політичного лідерства полягає в забезпеченні лідерами зв`язку як між масами й політичними інститутами, так і між самими політичними інститутами, у тому числі між очолюваними лідерами вищими органами держави — парламентом, урядом, главою держави, вищими судами. Завдяки лідерам відбувається координація та узгодження дій усіх суб`єктів політики.
\end{description}
\subsection{Поняття та сутність політичного лідерства в Україні}
За сім десятиріч була створена керівна верства, правляча еліта. На зміну їй після серпня 1991 року в Україні, як і в інших державах СНД, приходить нова правляча еліта, яка називає себе демократичною. Закономірно постає питання: чим вона відрізняється від старої і що у них спільного?

Відомо, що нова еліта виступала у виборчих кампаніях з гострою критикою корумпованості партократії, бюрократизму, штучного роздування штатів управлінців. Але коли сама перемогла (на республіканському, обласному, міському, районному рівнях), виявилось, що роздуті штати управлінців вона не скоротила, а, навпаки, значно розширила. Частина нової еліти використовує своє службове становище для збагачення, вступаючи, зокрема, в ринкові структури.

Слід відзначити, що багато західних політологів, застерігаючи проти різкої зміни еліт, використовує такий аргумент: стара еліта вже мала привілеї, перш за все майнові, нова ж представляє ті верстви населення, які обділені ними, роками відчували свою защемленність, а тому, перемігши, нова еліта прагне «взяти своє»

Сьогодні, як ніколи, все більше усвідомлюється необхідність моральної поведінки лідера в сфері політики. На перший план висуваються такі моральні якості, як совість, чесність і благородство. Для політичного лідера честь – це перш за все єдність слова і діла; благородство – це толерантність і повага до інакомислячих, у тому числі й до думки політичних опонентів.

При оцінці політичного лідера на перший план виступають наступні критерії: моральне обличчя, манера триматися під час спілкування з населенням, взаємодія з іншими політичними суб’єктами, висока політична культура. Не принижуючи значення вищеназваних критеріїв необхідно підкреслити, що оцінку харизматичному лідеру дає сама історична практика, причому оцінку неоднозначну. Тут виступає цілий ряд тенденцій: по-перше, можливість різкого і непередбаченого переходу до тоталітаризму в силу політичної могутності харизматичного лідера; по-друге, можливість ідеалізованої «реанімації» історичних персонажів, пов’язаних з міфологізацією особистості; по-третє, не завжди оправдана незалежність і зростання ролі інституту президента. Для ефективної боротьби з бюрократією, авторитаризмом, тоталітаризмом будь-яка система потребує харизматичного лідера.

Деякі політологи вказують, що основні тенденції формування політичного лідерства в Україні це – інституціоналізація, професіоналізація, зменшення вірогідності появи в сучасних умовах видатних політичних лідерів.
\subsection{Проблеми формування політичного лідерства в сучасній Україні}
Досить своєрідно трактує поняття "політичний лідер" відомий український політолог Д. Видрін.

Він вважає, що \textbf{політичний лідер} — це будь-який, незалежно від формального рангу, учасник політичного дійства, процесу, який намагається і спроможний консолідувати зусилля всіх, хто його оточує, і активно впливати (в межах території, міста, регіону, країни) на цей процес для досягнення висунутих ним визначних цілей.

Проблема лідерства завжди має розглядатися у контексті конкретного суспільства та історичного часу. Процеси, що відбуваються в Україні, значно відрізняються від тих, що здійснюються в інших посттоталітарних країнах Європи і колишніх республіках СРСР. Значна роль у цих процесах у нашій державі належить не правлячому класові чи будь-якій політичній силі, а саме політичному лідерові, особі як головній фігурі найважливіших політичних процесів. І це природно, оскільки Україна фактично завжди була суспільством лідерського типу. Князі, гетьмани, полководці, провідні політики здебільшого відігравали домінуючу роль у виборі шляхів суспільно-політичного розвитку країни.

Лідерством в Україні і сьогодні певною мірою компенсується брак усталених договірних норм, законодавчої бази, цілей та зв'язків, які були б визнані й прийняті до фактично обов'язкового виконання. Фахівці, вказуючи на брак розвинутої демократії в Україні, мають на увазі саме це.

Серед політичних лідерів сучасної України можна виокремити два типи:
\begin{description}
\item["поступливий"] — лідер консервативного типу, тобто такий, що під певним тиском намагається зберегти в країні раніше існуючу систему;
\item["інверсійний"] — лідер, якого визнають і сприймають через переслідування його владою чи критику з боку інших лідерів або політичних сил.
\end{description}
Аналізуючи феномен сучасного політичного лідерства в Україні, щодо багатьох лідерів абсолютно неможливо визначити, є вони лідерами загальнонаціональними чи регіональними; інтереси яких соціальних груп захищають; яке їх ставлення до існуючого суспільного устрою — функціональне, дисфункціональне чи стабілізуюче. Інакше кажучи, нині в Україні не бракує так званих "розмитих лідерів", які діють адекватно суспільно-політичним змінам і конкретним колізіям. Таких лідерів скоріше забагато. А тому поява дедалі більшої кількості лідерів елементарного популістського типу загрозлива для України. І це тоді, коли суспільство сьогодні надто гостро потребує сильних лідерів, гідних сповна взяти на себе відповідальність за долю усієї нації, а не лише за долю своїх виборців.

Безумовно, сучасна Україна потребує передусім демократичних лідерів, але багато з них все ще використовують авторитарні норми управління. Серед багатьох якостей сучасного політичного лідера в Україні варто виокремити почуття національної самосвідомості, відчуття єдності з народом. Інакше кажучи, лідера України характеризує не так уміння красиво висловлюватись, як те, що він здатен бути захисником національних інтересів, обстоювати їх.

Торкаючись проблеми політичного лідерства в сучасній Україні, можна сказати, що тут мають місце різні підходи. Деякі політологи вважають, що до політичних лідерів досить відносити тільки найвищих керівників держави, які очолюють верховну владу – Президент, голова уряду, Голова Верховної Ради. Інші вчені додають до цієї когорти лідерів політичних партій та рухів, найбільш відомих впливових міністрів, державних діячів іншого рангу, представників вітчизняного бізнесу. А є й такі (це можна часто побачити на сторінках газет), що до списку політичних лідерів відносять чиновників Адміністрації Президента, особливо його помічників, хоча всім відомо, що “лідерство” цих осіб обмежується їхніми службовими кабінетами та кріслами, з втратою яких зникає і все їхнє політичне лідерство.

Політична практика України періоду незалежності свідчить, що більшість лідерів загальнонаціонального рівня прийшли з регіонів, де вони опанували досвід регіональних керівників – В.Чорновіл (Львів), П.Лазаренко (Дніпропетровськ), В.Янукович (Донецьк), – або сформувалися як політики в регіональних умовах: Л.Кучма, О.Мороз, П.Симоненко, Ю.Тимошенко та інші .

Потреба в авторитетному лідері особливо зростає за складних ситуацій життєдіяльності суспільства. Можливості суспільства щодо розв’язання складних суспільних проблем багато в чому залежать від наявності загальновизнаного лідера, який не тільки запропонував би стратегію виходу з кризи, а зумів консолідувати суспільство на її виконання.

Президентські вибори в Україні часів державної незалежності свідчать, що на посаду загальнонаціонального лідера обиралися особи середнього рівня управлінського професіоналізму. Часто ці обирання були не висловленням народної любові або авторитету, а, скоріше, не бажанням бачити на цій посаді основного конкурента. Так було на президентських виборах 1991, 1994, 1999 і 2010 років. Винятком були тільки президентські вибори 2005 р., коли вирішальну роль відіграв зовнішній фактор.

Можна погодитися з висновком Д. Видріна й Д. Табачника, що перший склад політичного керівництва країни цілком заслуговує назви “вербальні лідери”. Прийшовши до влади за допомогою слова, вони свято повірили і абсолютизували силу друкованого і особливо усного публічного слова. Всю політику вони намагалися звести до пошуку потрібного дискурсу критики минулого, суперечки з опонентами чи розмови з виборцями. Але політичне лідерство неможливо звести до політичної декларації. Вочевидь, не лідери керували процесом становлення нових держав, а сам процес, який об’єктивно розпочався, визначив дії, погляди і вчинки цих лідерів.

Навмисна дистанція більшості загальноукраїнських політичних лідерів перших років незалежності від політичних партій та рухів сьогодні вже не виправдовується. І це зрозуміло, бо всі вони прийшли до влади не як вожді й борці за незалежність України, а випадково. Наявне падіння авторитету, різке зниження особистого політичного рейтингу примушує шукати політичну опору в політичних партіях і рухах.

В умовах сучасного пострадянського суспільства можна дійти такого висновку: найпоширеніші типи політичного лідера — це колишні партійні вожді і ватажки.

Але політична перспектива за іншою генерацією політичних лідерів, які зможуть опанувати і виховати в собі основні риси політичної культури, а також важливі психологічні дані: вміння працювати з людьми, комунікабельність, вміння обирати свою команду помічників та інше.

\textbf{Політичні лідери} – це люди, які мають бути професійно, інтелектуально, морально-психологічно та організаційно підготовлені до владної діяльності, спроможні справляти легітимний вплив на все суспільство або його частину.

Проте на українській політичний арені за роки незалежності, на жаль, так і не з’явилося особистостей, які були б цілком здатні вирішувати сучасні завдання не тільки розбудови державності, а й формування та функціонування інститутів громадянського суспільства. Навіть найвідоміші українські політичні лідери (державні діячі, керівники політичних партій, громадських організацій, суспільно-політичних рухів та об’єднань), спираючись лише на окремі соціальні верстви населення, не мають жодних підстав вважати себе авторитетними лідерами загальнонаціонального масштабу (хоча деякі політично заангажовані соціологічні центри та засоби масової інформації намагаються штучно створити декому з них сприятливий).

Розуміння елітою та політичними лідерами державної перспективної стратегії, формулювання базисних, системоутворюючих та зрозумілих для населення підходів і цілей – це не тільки завдання суспільної науки, а й еліт і політичних лідерів. Співпраця науки і політичних лідерів є основою успіху останніх. Просуватися вперед “навмання” – занадто велика розкіш і ризик для країни та її лідерів. І питання не тільки в тому, які цінності сповідує лідер, а ще й в тому, що треба знати, куди йти. Тільки наука дає політичним силам змогу мати довгостроковий авторитет і вплив, тільки науково визначена ідейно-теоретична основа дає змогу партії претендувати на представництво інтересів народу і перспективу стати владною. Тут зайві претензії на “єдино правильне вчення”. Тільки в діалозі політичних сил, у конкуренції ідей – гарантія становлення нових ефективних лідерів, ефективного розвитку країни.

Певні проблеми у процесі формування загальнонаціонального лідера в сучасній Україні вливають на привабливість держави. Отже, Україна втрачає привабливість як явище політичне, економічне і духовно-культурне. Розвиток країни, а саме незалежність, нові політичні інститути, заклики до духовного відродження перейшов у непристойне з політичної точки зору. Існують заклики повернути тоталітарне минуле, корупція владних структур, паплюження демократії, комерціалізація культури і занепад духу. Така ситуація має не тільки внутрішньополітичний і економічний ефект, але й міжнародний. Отже, падає довіра до України і української владної й опозиційної еліти як здатних до реформ, до позитивних перетворень. Міжнародні організації, фінансово і економічно заможні країни скорочують матеріальну підтримку нашій державі. Це свідчить, що падає довіра до владних еліт і політичних лідерів.

Таким чином, формування ефективних політичних лідерів є важливе суспільне завдання. Освоєння ними світового досвіду політичного менеджменту, вміння поєднувати світовий досвід управління з національними традиціями створює умови для становлення відповідальних і кваліфікованих лідерів.

Можна сказати, що, незважаючи на численні об’єктивні та суб’єктивні труднощі в процесі формування політичної системи держави, Україна створила певну законодавчу базу для становлення загальнонаціонального лідерства, до якого належать президенти, голови уряду, міністри, керівництво Верховної Ради, лідери впливових політичних партій та громадських рухів.

Створені, хоча і не зовсім демократичні, механізми формування регіонального адміністративно-управлінського лідерства, представниками якого є голови адміністрацій та органів місцевого самоврядування. Але, закономірно, напрошується і такий висновок, що сучасне українське політико-управлінське лідерство, як і вся політична система суспільства, перебуває в процесі реформування і має усі риси перехідного періоду.