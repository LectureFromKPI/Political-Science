\section{Вибори та виборчий процес}
Зміст лекции:
\begin{itemize}
	\item поняття виборів та характеристика їх видів та функцій
	\item абсентивізм та його причини
	\item виборчі системи, їх види та особливості функціонування
\end{itemize}

\textbf{Вибори} - політична процедура, що залучає маси до процесу управління державою. Ця процедура передбачає демократичний спосіб формування періодичної чи позачергової зміни персонального складу державних органів влади та підтвердження їх повноважень на певний термін, окреслення певного кола посадових осіб шляхом голосування за кандидатів, висунутих відповідно до встановлених законом вимог.

\textbf{Політичні вибори} - не лише процес голосування, а і широкий комплекс заходів щодо формування керівних органів держави. Головними етапами у підготовці та реалізації виборчого процесу є:
\begin{enumerate}
	\item призначення виду виборів(президентські/місцеві) та дати їз проведення(позачергові, або чітко визначені конституцією).
	\item визначення меж виборчих округів і виборчих дільниць - (бувають пропорційні та мажоритарні системи виборів. від цього залежить кільк виборчих дільниць)
	\item формування виборчих комісій (центральної, окружних, дільничих)  
	\item формування списків виборців
	\item висування та реєстрація кандидатів - відбувається за конституційними правилами.
	\item проведення передвиборчої агітації -  (новітні технології агітації)
	\item голосування та процес підрахунку голосів
	\item оприлюднення результатів голосування
	\item в разі необхідності проведення повторного голосування та повторних виборів.
\end{enumerate}

\subsection{Виборчі цензи. Абсентиїзм}
\textbf{Цензи} - заборони, що не дають змоги приймати участь у активному та пасивному виборчому процесі
\begin{itemize}
	\item стать
	\item вік (з 60-го року був знижений до 18 років) - у вищі палати від 30-40 років, президенство - від 35 років.
	\item майновий та соціальний ценз - часто присутніми на виборах дозволялося бути тільки у людей з певними майновими можливостями.
	\item ценз осілості (наприклад, президентом Америки може бути тільки уродженець, який прожив не менше 14 років у ній).
\end{itemize}
\textbf{Абсентиїзм} - явище (відсутній - лат.) свідомого ухилення виборців від участі у голосуванні. Причинами абсентиїзму можуть бути:
\begin{itemize}
	\item особливоті виборчої компанії
	\item певні причини особистого характеру(хвороба)
	\item загальна нецікавість або політична, соціально-економічна ситуація в державі
\end{itemize}
\subsection{Види виборчих систем}
Існують дві основні системи виборів. Під виборчою системою розуміється сукупність встановлених законом правил проведеня виборів, регламентом здійснення конкретних процедур виборчої кампанії визначенням результатів голосування, структурою голосування та виборчими формулами, які застосовуються.

Виборчі системи є різними модифікаціями двох основних типів виборчих систем
\begin{itemize}
	\item мажоритарної (орієнтована на конкретного кандидата)
	\item пропорційної
\end{itemize}
Ми знаходимося у змішанів виборчій системі (президент - мажоритарна, )
\subsubsection{Мажоритарна}
Характеризується 2-ма шляхами проведення : абсолютної більшості(переможесь 50+1 голос) та відносної більшості - застосовується при більшої кількості кандидатів.
\subsubsection{Пропорційна}
Застосовуються формули, що встановлюють, чи допускаєтсья якась партій до входу. При цьому встановлюється порог 5\% - для того щоб пройти в парламент. 
Включається система жорстких на нежорстких списків партій.
\subsubsection{Змішана}
Застосовується у Німеччині, Росії. Також Україна повернулася до змішаної системи, хоча далі не відомо, що буде.

 
