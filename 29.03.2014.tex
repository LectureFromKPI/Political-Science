\section{Особа та політика} 
План лекції:
\begin{itemize}
\item Особа та політична діяльність;
\item Сутність та основні чинники формування політичної культури та політичної свідомості;
\item Зміст та сутність політичної соціалізації.
\end{itemize}
Поняття \textbf{особа} та \textbf{особистість} позначають індивіда як суб’єкта відносин і свідомої діяльності та його соціальне та психологічне обличчя.

\textbf{Політична поведінка} - це сукупність реакцій соціальних суб’єктів (осіб, груп, спільнот) на певні форми, засоби і напрямки функціонування політичної системи.
Види політичної поведінки:
\begin{itemize}
\item Відкрита поведінка - це політична дія;
\item Закрита поведінка - це політична іммобільність.
\end{itemize}
Форми закритою політичної поведінки:
\begin{itemize}
\item Повна політична виключеність;
\item Політична виключеність як результат заорганізованості політичної системи;
\item Політична апатія;
\item Політичний бойкот.
\end{itemize}

\textbf{Політична дія} - це будь-яка відкрита поведінка (може спостерігатися) демонстрована в рамках політичної системи або сфери індивідом чи соціальною групою.

\textbf{Політична діяльність} - це складова людської діяльності, яка полягає в спрямованості на реалізацію політичних інтересів суб’єктів політики і насамперед на завоювання, утримання та реалізацію влади.

\subsection{Політична свідомість і політична культура як важливі поняття характеризують суб’єктивні аспекти політики}
\textbf{Політична свідомість} (сутність поняття) - результат і процес засвоєння політичної реальності з урахуванням інтересів людей.
Етапи політичної свідомості:
\begin{itemize}
\item Формування потреби;
\item Прийняття соціальних ролей;
\item Формування цілісних орієнтацій;
\item Ідентифікація зі своєю соціальною групою;
\item Усвідомлення своїх цілей і інтересів у політичній сфері.
\end{itemize}

\textbf{Політична культура} - це система цінностей соціуму, сукупність переконань, поглядів, ідей, установок, систему політичних інститутів і відповідних способів колективної та індивідуальної діяльності.

Функції політичної культури:
\begin{itemize}
\item пізнавальна;
\item регулятивна;
\item нормативно-цілісна;
\item захисна;
\item прогностична;
\item інтегративна;
\item комунікативна;
\item виховна.
\end{itemize}

Елементи політичної культури:
\begin{itemize}
\item політичні уявлення;
\item політичні переконання;
\item політичні традиції;
\item політичні установки;
\item політичні орієнтації.
\end{itemize}

Класифікація політичної культури за рівнем спільності:
\begin{itemize}
\item головнуюча
\item субкультура
\end{itemize}


