\section{Харектеристика основних соціально-полятичних підходів щодо шляхів трансформації українського суспільства на сучасному етапі} 
\textbf{Процес трансформації} – це складне переплетення стихійних та свідомих процесів. Це дії не лише елітних груп та носіїв інноваційної діяльності, але й стихійні процеси масової поведінки, соціальні наслідки реформ тощо. Перехід до нових видів відносин залежить не лише від рівня соціальної зрілості, активності еліти, але й від рівня соціальної свідомості, культури мас людей, від їхнього ставлення до реформ, до власного становища, форм діяльності.

На сьогодні \textbf{дослідницькою метою} соціологів виступає методологічне обґрунтування зовнішніх і внутрішніх чинників структурних перетворень в українському суспільстві, а також аналіз передумов, закономірностей, тенденцій і особливостей протікання трансформаційного процесу.

До \textbf{внутрішніх чинників} можна віднести:
\begin{itemize}
\item тимчасову протяжність соціальних процесів;
\item управлінську дію;
\item інформаційно-пізнавальний рівень суспільства і т. п. 
\end{itemize}

\textbf{Зовнішнім чинником} структурних перетворень виступає трансформація соціального простору. 

Для \textbf{трансформаційного суспільства характерн}е різке посилення індивідуалізації особистих життєвих практик, ослаблення їхньої залежності від приналежності людей до великих соціально-професійних формалізованих груп. 

Сучасні трансформаційні процеси в українському суспільстві супроводжуються соціально-структурними перетвореннями, формуванням нових верств, груп, зміною місця у стратифікаційній системі вже існуючих та зміною соціальної ідентифікації.

Основний \textbf{вектор трансформування політичної системи} в Україні, як і очікувалось від більшості пострадянських країн, це перехід від цілком твердо встановленого конкретного тоталітарного режиму, який є основним системоутворюючим чинником однойменної системи, "до цілком не визначеного" - нового демократичного режиму, у якому реформуванню мало б підлягати все—політичні інститути, відносини, політична культура.

У політично-правовому сенсі \textbf{реформи} (трансформації) — це поступова воле-встановлена зміна політичного устрою та законодавчої бази функціонування політики, яка, однак, не торкається фундаментальних основ суспільно-політичного життя. Такими фундаментальними принципами нашого політичного життя є визнання народу єдиним джерелом влади, республіканізм, соборність, унітарна держава.

Конкретний зміст \textbf{політично-правової реформи} політичної системи в Україні складається з таких напрямків:
\begin{itemize}
\item політична реформа:
\item конституційна реформа;
\item адміністративна та судова реформи;
\item реформа розподілу владних повноважень між головними суб'єктами державної влади.
\end{itemize}
\subsection{Висновок}
Соціальне життя в Україні висвітлюється в контексті розвитку соціального життя інших країн, яке на сучасному етапі характеризується двома основними рисами. По–перше, це життя отримало характер вільної, невпорядкованої соціальної течії, де в силу іманентних імпульсів виникають розпадаються різноманітні спільноти. Це життя звільнилось з-під політико – регламентуючого впливу, втратила свій політико – легітимний характер. По-друге, всі соціальні зв’язки та стосунки мають риси розвинутої індивідуальності, особистої незалежності людини.

Вивчення соціальної структури та соціальної стратифікації, тенденцій їх розвитку становить не тільки науковий, але й практичний інтерес, бо дозволяє визначити оптимальні для суспільства напрямки прогресу. Зокрема, нині в українському суспільстві актуалізується проблема запобігання його розвитку по шляху формування двополюсного закритого суспільства і переходу до відкритого суспільства з масовим та впливовим “середнім класом”, яке розглядається як соціальна база плюралізму та демократії, створює передумови для самореалізації особистості у бажаних для суспільства галузях. 